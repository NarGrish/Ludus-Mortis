\chapter{Das Spiel}
\section{Grundlagen des Spiels}
	
    \textit{Ludus Mortis} zählt als Pen and Paper zu den kooperativen Gesellschaftsspielen. Die Spieler übernehmen die Kontrolle über monströse Kreaturen, welche die Spieler im Laufe des Spiels immer mehr anpassen können, während der Erzähler sie durch eine einzigartige Geschichte führt. Im Unterschied zu der Mehrzahl an Pen and Paper werden in diesem Spiel handelsübliche Spielkarten anstelle von Würfeln genutzt. Für das Spielen werden daher folgende Utensilien benötigt:

    \begin{itemize}
	    \item Bleistifte und Radiergummi
	    \item Ein Kartendeck (2 bis As) für jeden Spieler und ein weiteres Gruppendeck (Gegnerdeck)
	    \item Ein Charakterprofil pro Spieler
	    \item Marker wie Münzen, Miniaturen, Effektkarten o.Ä.
	    \item Ein Maßband
    \end{itemize}
	
%===========================================================================%	
	
\section{Der Spielmeister}
    Damit die Spielergruppe in ihr Abenteuer starten kann, benötigt sie einen Spielmeister. Dieser Erzähler ist die mächtigste Instanz in dem Spiel: Er kreiert die Szenarien, kontrolliert tausende von Kreaturen und entscheidet grundlegend was in eurer Geschichte passiert. Zudem fällt ihm in den meisten Spielergruppen der organisatorische Teil zu und er ist der Einzige der das gesamte Regelwerk gelesen haben muss.
    
%===========================================================================% 

\section{Die Goldene Regeln}
    Für ein angenehmes und konfliktfreies Spiel benötigt es ein paar übergeordnete Regeln:
    \begin{enumerate}
         \item Sonderregeln, Effekte o.Ä. die den Grundregeln widersprechen, stehen stets über den Grundregeln. Sie existieren um gewisse Besonderheiten zu verdeutlichen und so den Spielern die Charakteristiken der Welt näher zu bringen.
         \item Sollte nichts anderes geschrieben stehen, haben Zahlenkarten den Wert der auf ihnen steht. Der Bube entspricht 11, die Dame 12, der König 13 und das Ass 15.
         \item Das Wort des Spielmeisters steht stets über jeder Regel.
    \end{enumerate}

%===========================================================================%    
    
\section{Das Spielprinzip}
    Durch die Abwesenheit von fortlaufenden Spielrunden oder Zugreinfolgen ist die Spielwelt frei zugänglich. In dem die Spieler ansagen welche Aktionen sie durchführen und im Spiel miteinander kommunizieren, können sie sich leicht mit ihrem Charakter identifizieren und alles tun was ihrem Monsterherz beliebt. Sie können sich frei auf der Karte bewegen, neue Regionen erkunden und dabei in die Geschichte des Erzählers eintauchen. Um Aktionen durchzuführen, spielen die Spieler Karten aus, vergleichen ihre Profilwerte und setzten geschickt ihre Fähigkeiten ein.
    
%===========================================================================%  

\section{Die heimliche Welt}
    Die heimliche Welt, ein großer Kontinent voller Leben! Von den trockenen Wüsten der Berstlande, entlang den Ufern der Naht, über die Gipfel der finsteren Gebirge bis hoch in die eiskalten Inseln Nurcarts: Ein jede lebende Seele dieser Welt weiß, was es heißt um sein Leben zu kämpfen. Sei es im Kampf gegen die plündernden Klans der Ghule, auf der Suche nach dem letzten fressbarem Korn im eisigen Winter oder aber das Korruptions überwucherte Leben in einer der großen, finsteren Städte. 
    Die heimliche Welt, ein herber Kontinent voller Unterschiede! Es sind nicht die Gefahren, die die Regionen der heimlichen Welt besonders machen - es sind die Einwohner, die Umgebung und die Hoffnungen. Dieser Kontinent umfasst hunderte von intelligenten Arten, tausende interessante Orte und Milliarden von Optionen. Der Süden geprägt von Spuren der alten Welt ist ein furchtbar heißer Ort. Einige Teile dieser Wüstenregionen liegen sogar so weit südlich, dass die Sonne Verala das gesamte Jahr lang nicht untergeht. Blickt man nun nördlich, so sieht man fruchtbareres Land. Wo im Osten die Eckbucht das Leben der Bewohner maßgeblich beeinflusst, stellt im Westen der gar riesige Fluss Nahrt das Zentrum von Handel und Gedei dar. 
    
\section{Das Zeichen}
Jede Kreatur in der heimlichen Welt wurde mit seiner Geburt von Nar Grish gezeichnet. Auch wenn diese Instanz nicht einmal den Glaubensbemühten Krograg ein Begriff ist beeinflusst diese Zeichnung einen jeden. Jede Art bringt unterschiedliche Ausprägung der Zeichen mit sich, die entscheiden wie die Kreatur durch die einzelnen Zeichen beeinflusst wird. Eine Kreatur kann entweder durch das Zeichen \textbf{Herz} $\Herz{}$, \textbf{Pik} $\Pik{}$,  \textbf{Karo} $\Karo{}$ oder \textbf{Kreuz} $\Kreuz{}$ gezeichnet sein. Jedes Zeichen verfügt auch über eine Gegeninstanz, welches die Schwäche des Zeichens darstellt.\\
\\
Das Zeichen \textbf{Herz} $\Herz{}$ lässt Geschick und Gewandtheit in Kreaturen aufblühen, was sie tendentiel zu hinterhältigen und präzisen Schattenkreatur macht. Die Schwäche einer Herzkreatur ist \textbf{Pik} $\Pik{}$.\\ 
\\
Das Zeichen \textbf{Pik} $\Pik{}$ verleiht seinen Kreaturen ein besonders starkes Gewebe, das nicht nur zum einfacheren Einstecken starker Angriffe, sondern auch zu unvorstellbaren Muskelkräften führen kann. Sie unterliegen jedoch \textbf{Karo} $\Karo{}$.\\
\\
Das Zeichen \textbf{Karo} $\Karo{}$ verbindet Kreaturen stark mit dem Spiritfeld (ein Plantenfeld, welches die heimlichen Lande mit mächtigen Schläge durchströmt) und gewährt ihnen ein übernatürliches Spiritpotential. Sein größte Schwäche stellt \textbf{Kreuz} $\Kreuz{}{}$ dar.\\
\\
Das Zeichen \textbf{Kreuz} $\Kreuz{}$ belohnt seinen Träger mit einem ausgezeichneten Waffengespür und taktischer Überlegenheit. Es schwächelt lediglich gegenüber \textbf{Herz} $\Herz{}$.\\
\\
Das Zeichen dient sowohl als erzählerische, wie auch als spielmechaniche Komponente. Einen zusätzlich wichtigen Effekt bringt zudem dass As deines Zeichens. Sobald dieses auf den Ablagestapel wandert wird es, nachdem die Handlung abgeschlossen wurde, Zeit den Ablagestapel zurück in dein Deck zu \textbf{mischen}. Solltest du eine Karte von deinem leeren Deck ziehen müssen, ohne das As aufgedeckt zu haben mischst du ebenfalls dein Ablagestapel zurück als nun neues Deck. Nähere Informationen zu anderen Effekten folgen in den Abschnitten der Charaktererstellung und Proben. 


    
%===========================================================================%

%Ende des Kapitels