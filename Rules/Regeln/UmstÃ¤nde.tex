\chapter{Umstände}

\section{Merkmale}

\subsection*{Element} \label{ef:element}
Die Unterschiedlichen Elementmerkmale wie Tod oder Frost stellen die verschiedenen Zweige der Spiritmaninfestionen dar. Sammelt der Spirituist höhere Ränge in seinem Element so werden ihm weitere Manifestionen oder mächtigere Ausführungen dieser zu gängig.

\subsection*{Rüstung} \label{ef:ruestung}
Um sicht zu Schützen haben viele Kreaturen Rüstungen vom tragen einer handwerklich gebauten Rüstung bis hin zum verdichtetem Fell ist alles zu finden. Rüstungen reduzieren den Schaden den eine Kreatur erleidet um ihren Rüstungswert, noch vor dem Verteidigen des Schadens. Es wird nur physicher Schaden reduziert.

\subsection*{Schwimmer} \label{ef:schwimmer}
Es gibt einige Kreatur in den Weiten der Wildniss die von Geburt an talentierte Schwimmer sind. Diese Kreaturen fühlen sich im Wasser sicher und müssen nur in Ausnahmesituationen Proben zum Schwimmen ablegen.

\subsection*{Toxin} \label{ef:toxin}
Der Toxinwert einer Kreatur gibt an, wie gefährlich ihre körpereigenden Toxine sind. Der Toxinwert gibt Boni auf bestimmte Effekte und Fähigkeiten.

\break
%  = = = = = = = = = = = %

\section{Begünstigungen}

\subsection*{Besserung} \label{ef:besserung}
Wenn eine Kreatur eine Besserung in Höhe von \textbf{X} hat, verringert sie am Ende der Kampfrunde ihren Wundenwert um \textbf{X}. Um es einfacher zu halten können Besserungen mit Blutungen verrechnet werden. Die Besserung zählt als Heileffekt und macht eine Kreatur mit dem unterschreiten des Wundenlimits durch Besserung wieder Kampffähig. 

\subsection*{Vorteil} \label{ef:vorteil}
Sich einen Vorteil über den Gegner zu verschaffen ist immer hilfreich.
Ein Vorteil kann genutzt werden um dir einen Bonus von +3 auf deine nächste Aktion zu geben. Dabei wird der Vorteil verbraucht. Falls du mehrere Vorteile hast gewährt jeder weitere Vorteil nach dem ersten einen weiteren Bonus von +1 auf die Aktion. Vorteile gehen am Ende der Kampfrunde verloren.

%  = = = = = = = = = = = %

\section{Leiden}

\subsection*{Blutend} \label{ef:blutend}
Einige Aktionen lassen eien Kreatur \textit{Bluten}. Eine blutende Kreatur hat über stetigen aber nachlassenden Blutverlust zu klagen. Um dies im Spiel wiederzuspiegeln erleidet eine Kreaturm pro Blutung am Ende der Kampfrunde eine Wunde und verringert die Blutung um 1.Wenn eine schon blutende Kreatur weitere Blutungen erleidet werden die Blutungen zusammenaddiert und ergeben den neuen Blutungswert. Durch blutungen erzeugte Wunden können keine schwere Verletzung zur Folge haben, auch nicht bei überschrittenden Wundenlimit.  

\subsection*{Brennend} \label{ef:brennend}
Eine Kreatur die in Brandgesteckt wurde erleidet starke Schmerzen, die nicht selten tödlich ausfallen können. Eine Kreatur mit \textit{Brennend} (X) verliert am Ende seiner Aktionsphase physische Wunden in Höhe seines aktuellen Brennenwertes. Anschließend reduziert der Spieler den \textit{Brennend}-Wert um eins. Sobald dieser Wert auf Null sinkt ist die Flamme erloschen. Die Flamme erlischt ebenfalls sollte es der Kreatur gelingen sich mit Wasser zu löschen. Für die Dauer des Umstands, kann die Kreatur sich in keinster Weise tarnen und erhellt dunkle Umgebungen. Falls die Kreatur das Opfer von \textit{\nameref{ef:gefroren}} wird, so dominiert der höhere der beiden Werte mit der Differenz der beiden Effektwerten. \\
\textbf{Bsp.:} \textit{Sasra leidet nach einem feurigen Kampf unter dem Umstand \nameref{ef:brennend} (3). Als sie panisch durch den Raum rennt wird sie Opfer eines Froststrahl der ihr den Umstand \nameref{ef:gefroren} (4) auferlegt. Da der Wert des Gefrierens größer als der des \nameref{ef:brennend}s ist, brennt sie zwar nicht mehr, dafür leidet Sasra von nun an aber unter \nameref{ef:gefroren} (1).}
Sollte ein weitere Brennenumstand die Kreatur treffen so wird der höhere Wert verwendet.

\subsection*{Erschöpft} \label{ef:erschoepft}
Wenn eine Kreatur erschöpft ist entfällt ihre nächste freie Aktion, danach ist sie nicht mehr erschöpft.

\subsection*{Gefroren} \label{ef:gefroren}
Wenn man eine Unterkühlung schon fluchend umgeht, so ist der Umstand \textit{Gefroren} überhaupt nichts für dich. Für die Dauer dieses äußerst frostigen Umstands erleidet die Kreatur jedes mal wenn sie Wunden verliert eine weite Menge an Wunden die ihrem \textit{Gefroren}-Wert entspricht. Wie bereits in \textit{\nameref{ef:brennend}} beschrieben gleichen sich die beiden Umstände aus. Sollte ein eingefrorener Charakter erneut \textit{gefroren} werden, so gilt der größere der beiden Werte. Den geringeren Wert verliert die Kreatur an Wunden (Ohne dass der \textit{\nameref{ef:gefroren}}-Effekt ausgelöst wird).

\subsection*{Geschockt} \label{ef:geschockt}
Eine Kreatur die Opfer eines Schocks wurde hat eine verzögerte Reaktionszeit und kann nur schlecht auf Angriffe Reagieren. Für die Dauer des Schocks darf die Kreatur, wenn sie angegriffen wird, keine Karte zu verteidigung aufdecken oder ausspielen. Am Ende seiner Aktionsphase reduziert der Spieler den Schockwert um eins bis dieser auf Null fällt und er wieder reagieren kann. Sollte ein mechanisches Wesen geschockt werden kommt es zu einer totalen Überladung der Systeme. Es erleitet 5 mal so viele Wunden wie der Schockwert beträgt, kann für diese Kampfrunde gar nicht handeln und verteidigen, ist danach jedoch wieder voll Einsatz bereit. 

%  = = = = = = = = = = = %

\section{Status}

\subsection*{Binden und Gebunden} \label{ef:binden_gebunden}
Wenn du eine Kreatur an dich bindest gilt diese als gebunden (an dich).     
Gebundene Kreaturen können nur die Kreatur als Ziel wählen an die sie gebunden sind. Eine Kreatur kann zwar mehrere Kreaturen binden, aber eine Kreatur kann immer nur an eine Kreatur gebunden sein, hierbei gilt die letzte Kreatur die dich gebunden hat.


\subsection*{Fliegend} \label{ef:fliegend}
Die meisten Kreaturen können nur davon träumen ihre Füße vom Boden zu lösen und frei durch den Himmel zu gleiten. Fliegende Kreaturen haben abhängend von ihrer Flughöhe verschiedene \textit{\nameref{ef:fliegend}}-Werte. Hierbei entspricht ein Flughöhe grob einem Meter Höhe vom Abflugpunkt gemessen. Während der Bewegung darf ein Meter Bewegung entweder horizontal oder in Form von Flughöhen genutzt werden, solange die Kreatur dabei nicht vom Boden abhebt. Um \textit{\nameref{ef:fliegend}}-Werte zu markieren solltet ihr überlegen statt Markern vielseitige Würfel oder andere direkt ablesbare Zähler zu nutzen. Doch wie äußert sich das fliegen im Kampf? Grundlegend befinden sich Kreaturen nur dann im Nahkampf mit einer fliegenden Kreatur (und andersherum) Wenn sie maximal ein Meter über der Kreatur fliegt und somit Opfer von den meisten Nahkampfangriffen werden kann. Für alle weiteren Belangen ist der Nahkampf genau so wie bodenständige Kämpfe. Fernkämpfe hingegen unterliegen genau so wie Volumen oder Flächeneffekte einer Veränderung: eine Kreatur die hoch über dir fliegt wird wohl kaum Opfer deiner Explosion am Boden sein. Um Pythagoras Rechnungen zu umgehen (wir verbieten das echten Nerd-Gruppen natürlich nicht) wird zur Bestimmung von Reichweiten entweder der horizontale Abstand oder der \textit{\nameref{ef:fliegend}}-Wert genommen, je nachdem welcher der beiden Werte größer ist. Flächeneffekte haben bloß Einfluss auf Kreaturen auf Flughöhe 1 und Volumeneffekte werden im Allgemeinen durch den Spielmeister bestimmt. Wenn sollte ein fliegende Kreatur unter \textit{\nameref{ef:gefroren}} oder \textit{\nameref{ef:geschockt}} leiden, so verlieren sie jede Runde an dreimal so viel Höhe wie der Umstandswert angibt.\\
Fliegende Kreaturen zählen \textbf{immer} als Bewegt!




