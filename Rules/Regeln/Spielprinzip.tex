\chapter{Das Spielprinzip}

\section{Proben}

Wann immer die Frage nach körperlichen oder geistigen Kompetenzen gefragt ist, wie z.B. das Klettern auf einen Baum oder das Einschüchtern einer Kreatur werden Proben relevant. Eine Probe testet ob du einer Tätigkeit gewachsen bist oder nicht, dazu musst du sie bestehen.

Eine Probe läuft folgendermaßen ab. Der Spieler erklärt was er für eine Tätigkeit ausüben möchte. Der Spielleiter entscheidet wie schwierig die Probe ist und welche Eigenschaft dafür benutzt werden sollte. Nun deckt der Spieler die oberste Karte seines Kartendecks auf und addiert den Kartenwert auf den Wert der benutzten Eigenschaft. Der sich ergebende Wert nennt sich Probenwert.

Um eine Probe erfolgreich zu bestehen muss der Probenwert größer oder gleich dem geforderten Wert der Schwierigkeit sein. Als Orientierung gelten folgenden Kategorien:

\begin{itemize}
    \item 12+ \textbf{ganz nebenbei:} \textit{Im Takt klatschen, leise sprechen}
    \item 18+ \textbf{gewöhnlich:} \textit{Energiekern einsetzen, Grabstein entfernen}
    \item 24+ \textbf{ungewohnt:} \textit{einen versteckten Menschen finden, drei Nork-Ale weghauen}
    \item 30+ \textbf{herausfordernd:} \textit{auf einem wütenden Drachenrücken festhalten, Daerisch tanzen}
    \item 40+ \textbf{heroisch:} \textit{einen Mrot zum lachen bringen, Unsichtbare sehen}
\end{itemize}

\subsection*{Gezeichnete Proben}
Eine Probe ist gezeichnet falls das Symbol der Karte, mit der die Probe abgelegt wird mit deinem gewähltem Zeichen oder deinem Gegenzeichen übereinstimmt. Eine gezeichnete Probe hat eine vorteilhafte Konsequenz, im Falle deines Zeichen oder eine nachteilhafte Konsequenz, im Falle des Gegenzeichens.
Dies entscheidet aber \textbf{nicht} über das bestehen der Probe.

\subsection*{Kritische Proben}
Als kritische Probe werden spezielle Proben bezeichnet in denen der Spieler eine weitere Karte aufdecken darf/muss. Der Idealfall ist das Aufdecken/Ausspielen einer Bildkarte deines Zeichens. In diesem Fall darfst du sofort eine weitere Karte von deinem Kartenstapel aufdecken und den Kartenwert hinzuaddieren. Der schlechteste Fall tritt ein sobald du eine Bildkarte deines Gegenzeichens aufdeckst. Sage sofort eine Farbe (Schwarz oder Rot) an. Decke ebenfalls eine weitere Karte auf. Sollte die Karte nicht deiner angesagten Farbe entsprechen geht deine Aktion furchtbar aus und die Karten zählen als Kartenwert null. Andernfalls legst du die zweite Karte einfach auf den Ablagestapel und verwendest deinen ursprünglichen Kartenwert. Der dritte, seltenere Fall ist das Aufdecken/Ausspielen eines Ases eines dritten Zeichens. In diesem Fall hast du die Wahl: verwendest du es als 15er Karte oder möchtest du eine kritische Probe ansagen? Im zweiten Fall sagst du wie beim Gegenzeichen eine Farbe an. Im Gegensatz zu dem alten Fall darfst du jedoch bei der korrekten Vorhersage die beiden Kartenwerte wie im optimalen Fall zusammen addieren. Es sei noch erwähnt, dass eine weitere kritische Probensituation aus der zählenden zweiten Karte entstehen kann die den Spieler vor die selbe Entscheidungen stellt und auf einer solchen Weise im normalen Spielfluss ein maximaler Karten von 96 erreichbar ist. \\

\subsection*{Überflüssige Proben}
Es wird im Spiel immer wieder Situationen geben in denen Spieler absolut triviale Handlungen ansagen werden und einige in denen sie das Unmöglich versuchen werden. Da Proben stets die Möglichkeit eines Erfolgs oder Misserfolgs beinhalten, sollten sie nicht von Spielern verlangt werden wenn das Ergebnis ihrer Handlung von vorne herein klar ist. So stellt die Probe eines Skrivas der versucht durch schnelle Schwanzrotationen zu fliegen keine Möglichkeit auf Erfolg dar und sollte bloß eine Probe erfordern falls der Spielleiter erfahren möchte wie iritierend diese Handlung auf alle Umstehenden wirkt. Im Allgemeinen sollten \textit{ganz nebenbei}- und \textit{heroische}-Proben nur eingefordert werden, wenn der Spielleiter einen konkreten geschichtsrelevanten Aspekt in ihnen sieht, um z.B. dem kreativen Krograg im finalen Endkampf die geringfügige Chance zu geben, seinen Gegner mit seiner mühsam ausgeklügelten Lüge in die Flucht zu schlagen.

\subsection*{Vergleichende Proben}
Wenn zwei Kreaturen gegen einander agieren so kann es sinnvoll sein die Proben der beiden Kreaturen mit einander zu vergleichen und einen Sieger zu kühren. Eine solche Probe wird \textit{vergleichende Probe} genannt. Die Proben hierbei nicht auf die gleiche Eigenschaft gehen sondern, sollen die aktuelle Situation am besten wiederspiegeln. Die Kreatur mit dem größeren Probenwert dominiert das Duel und gewinnt die Probe.

\subsection*{Proben im Kampf}
Sollte es dazu kommen das der Spieler im Kampf über einen schmalen Stab balancieren will wäre eine Probe auf Gewandtheit nötig. Dies Probe wird nicht normal abgelegt da der Spielercharakter unter besonderer Anspannung steht.
Die Probe wird als Aktion ausgeführt, dabei kann er entweder wie bei einer Aktion üblich eine Karte aus seiner Hand spielen oder eine Karte aus seine Hand ablegen und dafür die oberste Karte seines Decks nutzen. Den benötigten Wert legt wie auch bei Proben außerhalb des Kampfs der Erzähler fest.

% = = = = = = = = = = %

\section{Fähigkeiten}
Der Hauptunterschied zwichen Arten, aber auch Kreaturen der selben Art sind ihre Fähigkeiten. Diese machen jede Kreatur besonders und bieten ein einzigartiges Spielgefühl.
Fähigkeiten unterteilen sich dabei in Aktionen und Merkmale? (Passive).

\subsection*{Aktionen}
Aktionen sind aktive Fähigkeiten die durch das Spielen von Karten ausgelöst werden, dabei hat jede Aktion gewisse Anforderungen die erfüllt sein müssen um den Effekt der Aktion zu vollziehen. Eine Aktion kann nur gewählt werden wenn die Anforderungen erfüllt werden können, somit ist es nicht möglich das eine Aktion von sich aus fehlschlägt. Hauptsächlich finden Aktionen im Kampfgeschehen Verwendung. 

\subsection*{Passive}
Passive stellen gegenüber den Aktionen, die den aktiven Teil der Fähigkeiten abdecken den passiven Teil dar. Sie wirken solange sie nichts anderes Besagen durchgehend und sollten im Hinterkopf behalten werden. 


\section{Entwicklung}
Im Laufe der Lebens eines Charakters erlernt man viel nützliche Tipps und Tricks. Auch wenn die Art und Weise wie eine Kreatur lernt und sich weiterentwickelt ganz von ihrer Art abhängt so ist es für sie alle Entscheidung Erfahrungen zu sammeln. Mit genug Erfahrung steigen die Charaktere Gefahrenstufen auf und haben somit nicht bloß Zugang zu neuen Fähigkeiten sondern werden ebenfalls erfahrener im Umgang mit ihren bisherigen Aktionen. So steigt die Kenntnisstufe einer Fähigkeit mit einer gewissen Anzahl an Gefahrenstufen die als Kenntnisschwelle im Fähigkeitenprofil vermerkt ist. So würde ein Mrot der auf Stufe drei seine Fähigkeit \nameref{sk:schockladung} erlangt hat alle vier Stufen (also 7,11,15,...) seine Fähigkeit verstärken. Diese Fähigkeitskenntnis (kurz FK) äußert sich bei einigen Fähigkeiten über Effektveränderungen, aber im allgemeinen als Bonus auf deinen Kartenwert.

\begin{table}[h]
    \centering
    \begin{tabular}{c c}
        \begin{tabular}{c|c}
            Gefahrenwert & Erfahrung \\
            \hline
            \textbf{1.} & 0 \\
            \textbf{2.} & 400 \\
            \textbf{3.} & 900 \\
            \textbf{4.} & 1.600 \\
            \textbf{5.} & 2.500 \\
            \textbf{6.} & 3.600 \\
            \textbf{7.} & 4.900 \\
            \textbf{8.} & 6.400 \\
            \textbf{9.} & 8.100 \\
            \textbf{10.} & 10.000 \\
            \textbf{11.} & 12.100 \\
            \textbf{12.} & 14.400 \\
            \textbf{13.} & 16.900 \\
            \textbf{14.} & 19.600 \\
            \textbf{15.} & 22.500 \\
        \end{tabular} &
        \begin{tabular}{c|c}
            Gefahrenwert & Erfahrung\\
            \hline 
            \textbf{16.} & 25.600 \\
            \textbf{17.} & 28.900 \\
            \textbf{18.} & 32.400 \\
            \textbf{19.} & 36.100 \\
            \textbf{20.} & 40.000 \\
            \textbf{21.} & 44.100 \\
            \textbf{22.} & 48.400 \\
            \textbf{23.} & 52.900 \\
            \textbf{24.} & 57.600 \\
            \textbf{25.} & 62.500 \\
            & \\
            & \\
            & \\
            & \\
            & \\
        \end{tabular}
    \end{tabular}
\end{table}

Da auch die Widerstandsfestigkeit sich mit der Zeit steigert, erhöht sich der Wundenmaximum mit jedem neuen Gefahrenwert um die hälfte der obersten Karte. Ebenfalls steigt der Gift- und Psychemaximum um die höhe der jeweiligen Zehner stelle (unter 10 erhöht sich der Wert um 1).\\
\textit{Dein Charakter erreicht Gefahtenwert 3. Zuvor hattest du einen Wundenmaximum von 38, ein Giftmaximum von 21 und ein Psychemaximum von 13. Du ziehst ein Bube also eine 11, somit erhöhen sich deine Werte um 5 auf 43 beim  Wundenmaximum, um 2 auf 23 beim Giftmaximum und um 1 auf 14 beim Psychemaximum.}