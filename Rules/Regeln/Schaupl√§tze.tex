\chapter{Schauplätze}

\section{Geländetypen}
Ein jedes Umfeld wird durch seine großen und kleinen Feinheiten ausgeprägt. Um diese in das Spieleinfließen zu lassen gibt es verschiedene Geländetypen die besondere Vor- und Nachteile mit sich bringen. Zudem können den meisten bestimmte \textit{\nameref{geländebesonderheiten}} zugeschrieben werden die im folgenden Absatz erläuter werden.

\subsection*{Bäche und Flüsse}
\textbf{Besonderheiten:} \textit{\nameref{gb:unsicher_stand}},\textit{\nameref{gb:wasser}}\\
Flüsse und Bäche sind die Lebensadern der weiten, grünen Länder. Sie schlängeln sich viele Kilometer durch die Natur, vorbei an Dörfern und Städte. Solange das Wasserbett nicht zu tief ist, kann der Spielleiter Kreaturen die hinreichend groß sind erlauben das Wasser gehend zu durchqueren. Anderen Falls müssen die Kreaturen durch das Wasser schwimmen.

\subsection*{dichte Wälder}
\textbf{Besonderheiten:} \textit{\nameref{gb:unsicher_stand}}\\
Die alten Wälder des Südens und Westens weisen häufig eine besonders Hohe Dichte an Pflanzen auf. Es ist somit für große (und größere) Kreaturen nicht möglich zu fliegen. Zudem erhalten Fernkampfangriffe einen Malus von -2 Um die große Menge an Deckung und Sichtlinienstörer wiederzuspiegeln.

\subsection*{Hügel und Erhöhungen}
Erhöhte Positionen können in den meisten Gebieten gefunden und ausgenutzt werden. Gerade für Schützen und verzweifelte Verteidiger sind sie ein wertvolle Unterstützung. Fernkampfangriffe machen +2 Schaden und können auf ihre Reichweite die Hälfte ihrer ehemaligen Distanz addieren. Versuche gegen Kämpfer auf einer erhöhten Position zu kämpfern werden im Nahkampf mit einem Boni von +2 für den erhöhten Verteidiger dargestellt.

\subsection*{Furten}
\textbf{Besonderheiten:} \textit{\nameref{gb:unsicher_stand}}, \textit{\nameref{gb:wasser}}\\
Furten sind Stellen in Flüssen und Bächen die besonders gut zu überqueren sind. Sie sind häufig die Wurzeln von alten Städten und stellen einen guten Lagerplatz dar. Die Flachheit des Gewässers ermöglicht es den Reisenden die eine mittelgroße oder größere Statur besitzen das Wasser gehen zu durchqueren, was jedoch kein heiles Ankommen garantiert. 

\subsection*{Moore und Sümpfe}
\textbf{Besonderheiten:} \textit{\nameref{gb:wasser}}

\section{Geländebesonderheiten} \label{geländebesonderheiten}

\subsection*{Wasser} \label{gb:wasser}
Sei es ein See, das Meer oder die Kanalisation. Viele Orte sind sehr nass und zwingen die Landlebewesen dazu, sich Gedanken über das Schwimmen zu machen. Kreaturen die Wasserflächen durchqueren wollen und sie nicht überspringen oder überfliegen können/wollen sind dazu gezwungen zu schwimmen. So lange sie nicht über das Merkmal \textit{\nameref{ef:schwimmer}} verfügen müssen die Kreaturen in jeder Aktionsrunde in der sie nicht passen und sich im Wasser befinden oder sich schwimmend fortbewegen wollen eine Gewandheits- oder Kraftprobe (14+) absolvieren. Sollten sie scheitern erleidet die Kreatur 5 Wunden für jede Runde die sie bereits Schwimmproben nicht besteht. Ihre Aktion verfällt und der Spielleiter bewegt die Kreatur 2 Meter Strom abwerts. Schwimmende Kreaturen können keine Angriffe außer Schlagen durchführen. Der Umstand \textit{\nameref{ef:brennend}} verfällt restlos sobald sich die Kreatur ins Wasser bewegt.

\subsection*{Unsicherer Stand} \label{gb:unsicher_stand}
Rutschige oder instabile Oberflächen machen es Kreaturen schwer ihre Füße sicher zu setzen. Stationäre Verteidiger die sich auf unsicherem Grund stehen, werden wenn sie Wunden durch physische Angriffe erleiten zurückgedrängt. Wird eine Kreatur zurückgedrängt so bewegt sie und sich wenn möglich um einen Meter nach hinten. Der Angreifer darf wenn er möchte ebenfalls um einen Meter folgen falls der Angriff ein Nahkampf angriff war. Die Bewegungen können ggf. Reaktionsschläge von den Kreaturen auslösen die nicht der Angreifer/Verteidiger waren.

\subsection*{Überwuchert} \label{gb:überwuchert}
Alte Ruinen und naturüberlassene Waldgebiete tendieren dazu ein hinderlichen Pflanzenwuchs aufzuweisen. In überwucherten Gebieten zählt ein Meter Bewegung als sei die Kreatur 1.5 Meter gegangen.

\subsection*{Versinken} \label{gb:versinken}
Moore, Treibsand und andere Hungrige Untergründe stellen Gefahren für unvorsichte Reisende dar. Stationäre Kreaturen erhalten nach ihrer Aktionsphase (auch wenn sie gepasst haben) einen \textit{Versinkenmarker}. Dieser kann eine einfache Münze oder Würfel sein. Sobald die Kreatur sich erneut bewegen möchte muss sie eine Kraftprobe absolvieren. Der Wert den diese Probe erreichen muss setzt sich aus dem Gewicht der Kreatur und der Dauer des Aufenthalts zusammen. Auf einen Grundwert von 5 addiert sich pro Gewichtsstufe ein Wert von 3 auf die Probe (mittel = 6) und pro bereits stationär stehende Runde ebenfalls 3 auf den Wert. Eine schere Kreatur (9) die zwei Runden (6) stationär war muss also ein Wert von 20 Übertreffen um sich aus dem Morast zu ziehen. Nach der vierten Aktionsphase die eine Kreatur versinkt, gilt sie als versunken und verliert jede Runde 10 Wunden. Sie kann nicht mehr agieren und ist bewusstlos.p