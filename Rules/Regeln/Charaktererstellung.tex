\chapter{Charaktererstellung}


    \section{Kreaturenidentität}
        Das Spiel beginnt mit der Erstellung der Spielercharaktere. Dieser wichtige Schritt wird eure Kreaturen grundlegend für die nächsten Spielsitzungen formen und euch die Möglichkeit geben euch mit dem Charakter identifizieren zu können. Damit neue Spieler nicht von der Schieren Vielfalt erschlagen werden, ist es sinnvoll nach der Reihenfolge vor zu gehen und sich Zeit zum Erstellen der Charaktere zu nehmen. Sollte es schnell gehen müssen, so könnt ihr auf fertige Charaktere zurückgreifen.
        
    
%===================================================================================%

    \section{Schritt 1: “Deine Art”}
        Es gibt viele Arten dort draußen. Von brutalen Wareguards, über robuste Mrots und begabte Draekolin. Um die Reise raus in die Wildnis zu starten, solltet ihr erst einmal in euch reinhören und herausfinden was für ein Monsterherz in euch schlägt! Eure Art entscheidet über eure Gestalt, Fähigkeiten, Vor- und Nachteile. Jeder Spieler wählt eine Art und kann sich deren Besonderheiten durchlesen. Um den Fluss aufrecht zu erhalten ist es empfehlenswert die Profilinformation zu einem späteren Moment abzuschreiben, wenn die anderen Spieler mit anderen Dingen beschäftigt sind. Die Profile zu den Arten befinden sich zusammen mit Beschreibungen in Kapitel \verweis{ch:die_arten}, eine Kurzübersicht folgt:
        
        \begin{itemize}
	    \item \nameref{art:draekolin}
	    \item \nameref{art:ghul}
	    \item \nameref{art:krograg}
	    \item \nameref{art:mensch}
	    \item \nameref{art:mrots}
	    \item \nameref{art:skriva}
	    \item \nameref{art:wareguard}
    \end{itemize}
        
%===================================================================================%

    \section{Schritt 2: “Körpereigenschaften”}
        Der nächste Schritt um seinen Charakter zu erstellen ist die Bestimmung seiner physischen Werte. Hierbei handelt es sich um die Festlegung der Körpereigenschaften. Diese unterteilen sich in Robustheit, Gewandtheit, Spirit, Initiative, Kraft und Geschicklichkeit. Um die Höhe der einzelnen Werte der Charaktere zu ermitteln, durchlaufen die Spieler ein simples Verfahren. Der Spielmeister nimmt sich ein gemischtes Kartendeck von 7 bis Ass und legt zweimal so viele Karten offen auf den Tisch wie Spieler an dem Spiel teilnehmen. Also deckt er z.B. bei vier Spielern acht Karten auf. Daraufhin wählt jeder Spieler in Absprache mit seinen Mitspielern eine der Karten und legt diese vor sich ab. Da Spieler abhängig von ihrer Art für das Sammeln der verschiedenen Symboliken belohnt werden, kann es sich lohnen nicht bloß auf den Kartenwert zu achten.  Sind viele Spieler Teil der Erstellung kann es sinnvoll sein reihum die Karten zu wählen um einen chaotischen Start zu verhindern. Sobald sich ein Spieler eine Karte genommen hat, notiert er sich den Kartenwert unter der aktuellen Körpereigenschaft. Bei der ersten Karte stellt dies die Robustheit da. Habe alle Spieler eine Karte vor sich liegen, deckt der Spielleiter für jeden Spieler eine weitere Karte in die Mitte und die Spieler dürfen mit der nächsten Eigenschaft fortfahren. Diesen Prozess wiederholen die Spieler für alle der sechs Körpereigenschaften. 
        
        \subsection*{Robustheit}
            Nicht eine jede Seele auf Erden ist mit einem großen zähen Körper ausgestattet. Der Robustheitswert spiegelt die pure körperliche Fähigkeit dar, Schläge einzustecken und trotzdem weiter zu machen. Proben auf Robustheit werden meist abgelegt wenn die Kreatur z.B. von einem niederfallenden Felsbrocken getroffen wird.

        
        \subsection*{Gewandtheit}
            Bist du in der Lage durch die wilden Ruinen vergessener Zeiten zu tänzeln während du zwei dutzend Pfeilen ausweichen musst? Weicht deine Kreatur den scharfen Krallen des angreifende Skrivakultisten aus? Deine Chance auf Erfolg stehen deutlich höher wenn deine Kreatur sehr gewandt ist. Du wirst Beispiels weise deine Gewandtheit benötigen um Angriffen auszuweichen oder aber um auf dem Tanzball der Fürstin zu beeindrucken.

        
        \subsection*{Spirit}
            Das Spiritfeld ist ein neben dem Magnetfeld der Erde ein zweites Feld welches sich um die Erde erstreckt. Die Energie des Feldes, lässt sich manifestieren und umwandeln. Der Grad deiner Spiritfähigkeit spiegelt wider wie gut du im Umgang mit dem Spiritfeld bist. Du wirst diesen Wert unter anderem nutzen wenn es um die Nutzung von Artefakten oder das Anwenden von Manifestationen geht.

        
        \subsection*{Initiative}
            Ein Kampf kann häufig alleine dadurch entschieden werden, wer als erstes handelt. Deine Reaktionsgeschwindigkeit und Fähigkeit schnell Entscheidungen zu fällen, stellt die Initiative da. Bist du in der Lage schnell deine Klinge zu ziehen um einen Angriff zu parieren? Realisierst du die Gefahr die auf dich zu kommt, bevor sie dich überrascht? Eine Initiativprobe wird es dir sagen!

        
        \subsection*{Kraft}
            Auch wenn Schnelligkeit in einem Kampf von Vorteil sein kann, ein richtig harter Schlag im rechten Moment besiegt jeden Gegner. Deine Kraft stellt deine körperliche Stärke dar. Du wirst sie benötigen um Dinge auf zu stemmen oder aber deinen Gegner in die Dunkelheit zu schicken. 

        
        \subsection*{Geschicklichkeit}
            Wenn du vor hast haargenau eine Rune zu formen, eine Waffe abzufeuern oder besonders empfindliche Stellen des Gegners zu verwunden, so wird dich deine Geschicklichkeit unterstützen müssen. Ein jedes Mal dass du besonders fingerfertig sein muss wirst du eine Probe auf Geschicklichkeit ablegen dürfen.

    
%===================================================================================%

    \section{Schritt 3: “Restkarten”}
        Nun bleibt eine Anzahl an Karten übrig die mit der Spieleranzahl übereinstimmt. Diese Restkarten stellen für die Spieler eine Möglichkeit da ihre Körpereigenschaften einen letzten Schliff zu verleihen. Die Spieler entscheiden gemeinsam welche Spieler am welche Karten am dringendsten benötigen und teile sie darauf hin so auf, dass ein jeder Spieler nun eine siebte Karte vor sich liegen hat. Die Hälfte des Kartenwertes entspricht einem persönlichen Punktekonto, dass er in Schritt 5 der Charaktererstellung nutzen darf um seine Attribute zu verbessern. Zu diesen zählen nicht bloß die Körper- sondern auch die Charaktereigenschaften. Du darfst dabei kein Attribut über den Wert 16 erhöhen.

    
%===================================================================================%

    \section{Schritt 4: “Charaktereigenschaften”}
        Doch bevor ihr euer Punktekonto leer räumt wäre es ja interessant zu wissen wie eure Charaktereigenschaften eigentlich aussehen.Nun nimmt jeder Spieler die Werte der Körpereigenschaften und trägt diese in beliebiger Reihenfolge in seine Charaktereigenschaften ein.
        
        \subsection*{Intelligenz}
            Nicht jede Kreatur stürmt ohne Gedanken an sein eigenes Leben in verfeindete Gebiete. Es gibt viele die stattdessen einen Plan schmieden, Unparteiische bestechen oder über geschickte Ideen einen Weg um das Problem herum finden würden. Der Erfolg solcher Listen werden maßgeblich durch deine Intelligenz bestimmt.
            
        \subsection*{Charisma}
            Mit einem kleinen Lächeln, den richtigen Worten oder bloß wohl bedachten Gestik ist es nicht selten möglich gefahrvolle Scharmützel und Streitigkeiten aus dem Weg zu gehen. Solche sozialen Interaktionen werden einem Charakter leichter fallen sollte er ein gewisses charismatische Händchen haben.
            
        \subsection*{Emotionen}
            Emotionen können eine mächtige Waffe sein. Bleibt man in den rechten Situationen ruhig, motiviert man sich unter Druck und reitet sich selber in eine mächtige Rage, so kann man entscheidende Unterschiede machen. Wie gut der Charakter seine Emotionen in die richtige Richtung lenken kann wird durch seinen Emotionswert ausgedrückt.

        \subsection*{Kreativität}
            Seien es die imposanten Orkgebilde, die schmierigen Pläne der Skriva oder die blutigen Gemälde der Vampire, doch eine jede Art weiß sich mit ihr auszudrücken: die Kreativität. Sie ermöglicht neue Konzepte und Fortschritt, währenddessen sie auch bestaunt und beeindruckt.

        \subsection*{Sinne}
            Du bist im Wald auf der Suche nach Beute? Folgst den Spuren eines Mörders? Versuchst deinen Verfolgern zu entkommen? Nun mit ausgeprägten Sinnen wird es dir deutlich leichter Fallen deine Erfolgschance deutlich zu erhöhen. Für jegliche Proben die deine Wahrnehmung austesten werden deine Sinne dir helfen.

        \subsection*{Willenskraft}
            Es ist nicht schwer Kreatur das Fürchten zu lehren, schwer ist es der Furcht zu wieder stehen. Mit einem Ziel, einer Vision, einer hoffnungsvollen Erinnerung kann selbst die einsamste Kreatur, sich aufraffen und die Zähne zusammenbeißen um für eine groß Sache weiter zu machen. Diese eiserne Überzeugung wird durch die Willenskraft ausgedrückt.

    
%===================================================================================%

    \section{Schritt 5: “Letzter Schliff”}
        Jetzt dürfen die erlangten Attributserhöhungen genutzt werden um Körper- und Charaktereigenschaften zu verbessern.
        Abschließend wähl jeder Spieler sein Zeichen aus trägt die Grundwerte, Genetik, Aktionen, Eigenschaften und die erlangten Entwicklungen ein. Wichtig: Die Symbole der Karten bestimmen die Genetik deiner Kreatur und geben einen Bonus, welcher bei der jeweiligen Art zu finden ist. Nachdem der Spieler noch seine Erscheinungsmerkmale wie Gewicht und Geschlecht gewählt hat ist der Charkter fertig und regeltechnisch spielbereit.

    \section{Sprachen}
    
    \subsection*{Delch}
    Die Sprache der Zivilisation wird sie genannt. Das Monument der modernen Welt. Die Kreation der Menschen. Delchs direkte Grammatik, eindeutige Vokabeln und starke Aussprache bilden eine logisch aufgebaute Sprache die Sich im Süden und Westen der heimlichen Lande nun über die letzten drei Jahrhunderte ausbreitet. Sie stammt ursprünglich aus Dermisch, einer der drei Reiche die heute zusammengeschlossen unter dem Banner des Dolreichs ruhen.
    
    \subsection*{Drroka}
    So rau den Name der Sprache einem bereits über die Zähne kriecht so simpel ist die sie auch. Jedes Wort fasst eine Unmenge an Bedeutungen zusammen und lässt sich im Sachzusammenhang interpretieren. So kann \textit{grekka} Krieg, Kampf, Tod, Schmerz, Spaß oder Raub bedeuten und besitzt noch nicht einmal Konjugationsformen. Drroka wird von Ghulen und Mrots gesprochen. Einige Skriva haben sich den Spaß gemacht aus Drroka ein Trinkspiel zu kreieren so, dass man sich in den Nav'rifbergen in der Regel mit Drroka ein Getränk bestellen können sollte.