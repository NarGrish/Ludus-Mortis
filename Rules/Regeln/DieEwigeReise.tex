\chapter{Die Reise des Lebens}

Das Leben besteht jedoch nicht bloß aus Kämpfen und Sterben - zumindest solang du nicht gerade ein Witchgeschöpf bist, das in der Hitze der Schlacht geschmiedet wurde. Die Kreature der Spieler werden ebenso viel Zeit auf der Straße, in Läden und als Gast in Häusern verbringen wie sie kämpfen werden. In diesem Zeitraum liegt es viel an den Ideen und der Wortgewandtheit des Spielleiters die Spieler durch eine interessante, interaktive und vor allen lebendige Welt zu geleiten. Erst eine Emotionale Verantwortung in dem Umfeld eines Charakters wird ihn dazu antreiben sein Leben im Duell mit einem Drachen aufs Spiel zu setzen.

\section{Die Bewohner und Gäste}

\subsection*{Größenordnungen}
Jede Kreatur hat eine Größenordnung in die sie fällt. Diese sind notwendig um in Aktionen fest zu stellen ob man sich z.B. hinter jmd. verstecken oder eine Kreatur um stürmen kann. In absteigender Reihenfolge sind die Größenordnungen:

\begin{enumerate}
    \item \textbf{Gigantisch:} \textit{Tempeltürme}, \textit{\nameref{NPC:Kriegsläufer}}
    \item \textbf{Riesig:} \textit{ausgewachsene Bäume}
    \item \textbf{Groß:} \textit{Dorks}
    \item \textbf{Mittel:} \textit{Menschen}
    \item \textbf{Klein:}
    \item \textbf{Winzig:}
\end{enumerate}



\section{Reisen}
\section{Städte und Ortschaften}

\section{Freizeit und das Leben}
Besonders fesselnde Geschichten gehen Hand in Hand mit tiefen emotionalen Bindungen von Kreaturen. Der Söldner der Nach Jahren die Blutschulden seiner Familie abbezahlen kann und so mit seine Frau zurückgewinnt hat nicht bloß ein großen Potential für heroische Kämpfe sondern auch einzigartige erzählerische Momente.

\subsection*{Speisen und Trunk}
Um an Kräften zu bleiben muss eine Kreatur Nahrung und Flüssigkeiten zu sich nehmen. Bis auf ein paar wenige Ausnahmen wie Geistern überleben Geschöpfe keine langen Zeiten ohne Nahrungsaufnahme. Um ein umständliches Auflistsystem zu verwenden, setzt \textit{Ludus Mortis} auf eine Vereinfachung. Am Ende des Tages muss jede Kreatur sich eine Ration und eine Portion Zutrinken aus dem Inventar streichen. Dies ist nicht nötig wenn die Charaktere im Laufe des Tages in Gasthäusern gespeist oder zu einem anderen Zeitpunkt eine derartige Nahrungsaufnahme durchgeführt haben. Sollte eine Kreatur nicht trinken/essen können gilt sie von nun an als \textit{hungrig}/\textit{durstig}. Für beide Umstände gibt es jeweils einen Malus von -3 auf alle Eigenschaften. Sollte eine Kreatur bereits \textit{hungrig}/\textit{durstig} sein, ist sie nicht mehr in der Lage zu reisen oder sich zu regenerieren und darf sich nicht verteidigen und keine \textit{Angriffsaktionen} mehr durchführen. Sie gilt als \textit{verhungernd}/\textit{verdurstend}. Sollte sie nun ein weiteren Tag nicht essen/trinken stirbt die Kreatur. Sollte ein \textit{hungrige}/\textit{durstige}/\textit{verhungernde}/\textit{verdurstende} Kreatur Essen/Trinken aufnehmen wird der Effekt von ihr entfernt. Besondere Umstände wie Hitze, große athletische Tätigkeiten oder Ähnliches erlauben es dem Spielleiter schon nach einem Tag die Effekte \textit{verhungernd}/\textit{verdurstend} Kreaturen anzuhängen.

\subsection*{Ruhe und Schlafen}

\subsection*{Freundschaften und Verpflichtungen}

\subsection*{Romanzen, Liebe und Sex}
Nerds (also unsere primäre Spielergruppe) sind in der Regel nicht als Romantiker bekannt. Einige Spielergruppe meiden den Themenblock ganz, andere sehen das Pen \& Paper als Möglichkeit auch mal auf der anderen Seite zu stehen. In dem Sinne sollte sich die Spielergruppe einig sein in wie weit sie Romanzen und Sex in ihr Spiel einfließen lassen möchte. Keine Spieler sollten in unangenehme oder bloßstellende Situationen gebracht werden und alle sollten mit den verbal dargestellten Szenen einverstanden sein. Sei dies gesagt, so sollte jedoch auch erwähnt sein das Liebesgeschichten extrem schnell zu interessanten und denkwürdigen Momenten führen können und ein großes Storypotential bieten. Zu dem können vergewaltigende Plünderer moralische Gründe zu Feindschaften und Bordelle geheimnisvolle Orte für dunkle Machenschaften mit sich bringen.