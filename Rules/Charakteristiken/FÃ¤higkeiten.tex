\chapter{Fähigkeiten}

\section{Freie Aktionen} \label{freie_aktionen}

\subsection*{Laufen} \label{sk:laufen}
Eine jede Kreatur ist auf ihre Bewegung angewiesen. Sei es um Nahrung zu besorgen, einen Partner zu finden oder aber um nach Hause zu gelangen. In Kampfsituationen ist das Bewegen in vielerlei Hinsicht wichtig, kann jedoch das Ausführen von besonders schwerwiegenden Angriffen verhindern.\\
\textbf{Effekt:} Bewegen dich bis zu einer Distanz von Bewegung in Metern (Zoll), aber nicht weniger als 1 Meter.\\
\textbf{Status:} Dein Status ist Bewegend.

\subsection*{Stehen} \label{sk:stehen}
Es ist häufig besser stehen zu bleiben um mit besonderer Stärke oder Präzision seinen Angriff oder Paraden zu koordinieren.\\
\textbf{Effekt:} Bewege dich maximal um einen Meter. Wenn keine frei Aktion angesagt wird bleibt man ebenfalls Stehen\\
\textbf{Status:} Dein Status ist Stationär.

\section{allgemeine Fähigkeiten} \label{allgemeinskills}

\subsection*{Sprinten} \label{sk:sprinten}
Manchmal ist es klüger zu laufen als sich seinem Feind zu stellen.\\
\textbf{Grundwert:} Initiative \\
\textbf{Effekt:} Bewege dich bis zu deinem Bewegungswert in Meter (Zoll) und setze deinen \textbf{Status} auf \textbf{Bewegend}.

\subsection*{Standfest} \label{sk:standfest}
..........................\\
\textbf{Grundwert:} Initiative \\
\textbf{Effekt:} Bleibe stehen und stärke deine Verteidigung (Blocken) bis zu deiner nächsten Aktionsphase um FK + 3. \textbf{Status} auf \textbf{Stationär}.

\subsection*{Vorausplanen} \label{sk:vorausplanen}
Wenn man nicht weis was man als nächstes tun soll, kann man seinen nächst Schritt vorausplanen. \\
\textbf{Grundwert:} Intelligenz \\
\textbf{Effekt:} Ziehe zwei Karten, dann lege zwei Karte ab.

\subsection*{Never tell me the odds} \label{never_tell_me_the_odds}
Wenn man nicht weis was man als nächstes tun soll, kann man seinen nächst Schritt vorausplanen. \\
\textbf{Grundwert:} Kreativität \\
\textbf{Effekt:} Lege deine ganze Hand ab, um soviel Karten neu zu ziehen.

\subsection*{Blocken/Ausweichen} \label{sk:blocken/ausweichen}
..............\\
\textbf{Grundwert:} Robustheit / Gewandtheit \\
\textbf{Kenntnisschwelle:} 1 \\
\textbf{Maximale Kenntnis:} 30 \\
\textbf{Anforderung:} Stationär / Bewegend \\
\textbf{Effekt:} Verringert den erlittenen \textbf{Physischen} Schaden um Ges-Wert. Statt eine Karte zu Spielen kann du auch die oberste Karte des Stapel benutzen.

\subsection*{Giftresistenz/Mentaler Widerstand} \label{sk:giftresistenz/mentaler_widerstand}
..............\\
\textbf{Grundwert:} Robustheit / Willenskraft \\
\textbf{Kenntnisschwelle:} 1 \\
\textbf{Maximale Kenntnis:} 30 \\
\textbf{Effekt:} Verringert den erlittenen \textbf{Gift / Psyche} Schaden um Ges-Wert. Statt eine Karte zu Spielen kann du auch die oberste Karte des Stapel benutzen.

\subsection*{schwerer/schneller Schlag} \label{sk:schlagen}
Nicht alle Kämpfe werden durch mächtige Blitzeinschläge oder Salven von Bolzen entschieden. Die meisten sind das Niederringen des Gegners durch pure ungebändigte Kraft und Gewalt.\\
\textbf{Grundwert:} Kraft / Geschicklichkeit\\
\textbf{Kenntnisschwelle:} 1 \\
\textbf{Maximale Kenntnis:} 30 \\
\textbf{Anforderung:} Stationär / Bewegend\\
\textbf{Reichweite:} 0.75 m \\
\textbf{Effekt:} \textit{Angriff (Physisch)}.

\subsection*{Niederdreschen} \label{sk:niederdreschen}
Die großen senkrecht angebrachte Klinge der Sense eignet sich hervorragend um durch eine Fläche an Gegnern zu dreschen und somit Feinde zu treffen. \\
\textbf{Grundwert:} Kraft - 2 \\
\textbf{Kenntnisschwelle:} 8 \\
\textbf{Maximale Kenntnis:} 3 \\
\textbf{Anforderung:} Stationär\\
\textbf{Radius:} Waffenreichweite (Halbkreis)\\
\textbf{Effekt:} \textit{Angriff (Physisch)}. Es werden alle Kreaturen in dem gewählten Halbkreis Opfer dieses Angriffs. \textbf{Benötigt eine Sense oder Großschwert.}

\subsection*{Speerwall} \label{sk:Speerwall}
Es ist seltend ein wohlwolliges Gefühl auf eine bewaffneten Feind zuzustürmen. Wenn dieser jedoch Speere auf dich richtet ist es ein weit aus schlechteres.\\
\textbf{Grundwert:} Kraft \\
\textbf{Kenntnisschwelle:} 5 \\
\textbf{Maximale Kenntnis:} 4 \\
\textbf{Anforderung:} Stationär 18+ \\
\textbf{Effekt:} Erhalte einen Bonus auf deine nächste Verteidigung gegen ein physischen Angriff, der der positiven Differenz dieser Probe entspricht. Solltest du keine Wunden in dem Angriff erleiden, hast du die Möglichkeit selber eine Angriffsaktion Gegen den Angreifer aus zu führen. \textbf{Benötigt einen Speer.}

% = = = = = = = = = = = = %

\section{Fähigkeiten der Draekolin} \label{draekolinskills}

\subsection*{Blendender Glanz} \label{sk:blendender_glanz}
Draekolin sind äußerst listige Kreaturen, wenn es darum geht einen Vorteil im Kampf zu kreieren. Mit kleinen reflektieren Eiskristallen auf ihrem Fell sind sie in der Lage gezielt Feinde zu blenden. \\
\textbf{Grundwert:} Geschicklichkeit \\
\textbf{Kenntnisschwelle:} 3 \\
\textbf{Maximale Kenntnis:} 5 \\
\textbf{Anforderung:} $\Herz{}$ 18+ \\
\textbf{Reichweite:} 10 m\\
\textbf{Effekt:} Dein Ziel kann die nächsten 2 Aktionsphasen nur noch Aktionen bis 0.5 m ausführen. Dieser Effekt bleibt nicht zwichen Kampfrunden erhalten.

\subsection*{Eishart} \label{sk:eishart}
Der Bolzen war präzise geschossen, sauber gefertigt und schnitt problemlos durch die Luft. Es hätte keine Möglichkeit geben können diesen Schuss zu überleben. Und doch steht der kleine Krieger tapfer in seinem unheimlichen Schwall an Nebel ohne Schwäche zuzeigen. \\
\textbf{Effekt:} \textit{Passiv.} Du darfst ein As ablegen, um die Wunden, die du durch einer gegen dich gerichteten Aktion verlieren würdest, zu halbieren.\\
\textbf{Steigerung [15]:} Nun darfst du ein As ablegen, um eine Aktion gegen dich zu ignorieren.

\subsection*{Eisig} \label{sk:eisig}
In deinem Leben hast du dich nicht nur mit der Kälte abgefunden, sondern wohlwollend aufgenommen, das sie nun ein teil von dir ist.\\
\textbf{Effekt:} \textit{Passiv.} Solltest du das Leiden \verweis{ef:gefroren} haben erleidet du durch den Effekt kein Wunden sondern heilst dich stattdessen.

\subsection*{Eiskalter Blick} \label{sk:eiskalter_blick}
Ein raues Bauchgefühl lockt den Blick des hasserfüllten Kämpfers einige Meter zur Seite. Die kalten silberglänzenden Augen eines Draekolin fangen ihn auf und fesseln ihn. Diese Tiefe die man in diesen kleinen Augen finden kann ist wahrlich beeindruckend. Und tendenziell tödlich.\\
\textbf{Grundwert:} Willenskraft (- Willenskraft des Ziels) \\
\textbf{Kenntnisschwelle:} 7 \\
\textbf{Anforderung:} $\Herz{}$ 15+ \\
\textbf{Reichweite:} 2 m
\textbf{Effekt:} Das Ziel muss in den nächsten \textbf{FK} Aktionsphasen Bildkarten spielen um Aktionen durchzuführen. Dieser Effekt bricht ab, sobald das Ziel ein As spielt.

\subsection*{Eiskalter Hauch} \label{sk:eiskalter_hauch}
Der eiskalte Hauch ist wohl der gefürchteste Hauch des Kontinents. Er ist weitmehr als lediglich ein Hauch. Er ist eher ein Gruß der Kälte. \\
\textbf{Grundwert:} Spirit \\
\textbf{Kenntnisschwelle:} 4 \\
\textbf{Anforderung:} Stationär 18+ \\
\textbf{Reichweite:} 5 m\\
\textbf{Effekt:} Belege dich und bis zu \textbf{FK} +1 weitere Kreaturen mit dem Leiden \textit{\nameref{ef:gefroren}}.

\subsection*{Eismantel} \label{sk:eismantel}
Durch ein Geklimper angekündigt hüllt sich der Draekolin in einen majestätischen Mantel aus Eis. \\
\textbf{Grundwert:} Spirit \\
\textbf{Kenntnisschwelle:} 1 \\
\textbf{Anforderung:} Stationär $\Herz{}$ 10+ \\
\textbf{Effekt:} Beschwöre ein Eismantel um dich herum, welches die nächsten \textbf{FK} Wunden absorbiert.

\subsection*{Ins Exil} \label{sk:ins_exil}
Wenn der Richtspruch des Draekolin dich als unwürdig ausersehnt, so wirst du keine Wahl haben, denn der Fluss des Feldes folgt der Gnaden des Richters. \\
\textbf{Grundwert:} Spirit \\
\textbf{Kenntnisschwelle:} 3 \\
\textbf{Anforderung:} $\Karo{}$ 15+ \\
\textbf{Reichweite:} 10 m \\
\textbf{Effekt:} Das gewählte Ziel kann die nächsten \textbf{FK} Aktionsphasen keine Aktionen mit dem Grundwert Spirit durchführen.

\subsection*{Ruf der Ahnen} \label{sk:ruf_der_ahnen}
Das fahle Rufen der Ahnen ruft die Ungläubigen zurück zu dem Rächer des Frostes um sie der Wahrheit zu überführen.\\
\textbf{Grundwert:} Charisma \\
\textbf{Kenntnisschwelle:} 5 \\
\textbf{Anforderung:} Stationär $\Kreuz{}$ 15+ \\
\textbf{Reichweite:} 10 m \\
\textbf{Effekt:} Binde bis zu \textbf{FK} Gegner an dich.

\subsection*{Kälte} \label{sk:kaelte}
Mit einem weiten Schlag fährt die Stahlgeschmiedete Klinge nieder auf den Draekolin. In dem Moment des Aufpralls verstummt der Klang des Kampfes für einen Moment. Die Schemen von Eis huschen über die Klinge bis hoch in den Arm des überraschten Angreifers. Mit einm süffisanten Grinsen fährt der Draekolin herum. \\
\textbf{Effekt:} \textit{Passiv.} Falls eine Kreatur einen erfolgreichen Nahkampfangriff gegen dich ausführt, erhält sie nach ihrem Angriff das Leiden \textit{\nameref{ef:gefroren}}.

\subsection*{Sternenstaub} \label{sk:sternenstaub}
Mit einer weiten ausholenden Geste wirbelt der Draekolin einen goldenen Strahl an feinsten Partikeln, die als Sternenstaub bekannt sind, auf den unartigen Widersacher. Wie fesseln setzten sich die Trägenteilchen an die Haare und Haut und halten ihn zurück.\\
\textbf{Grundwert:} Spirit \\
\textbf{Kenntnisschwelle:} 5 \\
\textbf{Anforderung:} $\Karo{}$ 15+ \\
\textbf{Reichweite:} 5 m\\
\textbf{Effekt:} Makiere eine Kreatur. Diese kann \textbf{FK} Aktionsphasen lang nicht mehr die Aktion Ausweichen nutzen.


% = = = = = = = = = = = = %


\section{Fähigkeiten der Ghule} \label{ghulskills}

\subsection*{Berserkerrausch} \label{sk:berserkerrausch}
Mit einem wilden Gebrüll fällt das schwere Metall nieder auf die hilflosen Gegner. Immer und immmer wieder. Gnadenlos.\\
\textbf{Effekt:} \textit{Passiv.} Sollange wie du mehr als die Hälfte deiner maximalen Wunden erlitten hast werden deine Waffenboni verdoppelt.

\subsection*{Einschüchtern}
Eine riesige Kreatur, welche vor die Aufragt kann der Moral deiner Soldaten stark zu setzen und sie vielleicht sogar in die Flucht schlagen.\\
\textbf{Effekt:} \textit{Passiv.} Die Willenskraft von Kreaturen die im Nahkampf mit dir sind wird halbiert.

\subsection*{Erstschlag} \label{sk:erstschlag}
Mit aggressiven Sturmangriffen setzten Ghule Verteidiger unter extremen Druck und provozieren so schnell Verluste unter Gegnern. \\
\textbf{Grundwert:} Kraft \\
\textbf{Kenntnisschwelle:} 2 \\
\textbf{Anforderungen:} 15+ \\
\textbf{Reichweite:} 0.5 m oder Waffe  \\
\textbf{Effekt:} \textit{Angriff (Physisch)}: Falls es die erste Aktion in der Kampfrunde ist, wird der Kartenwert (Nah und Fernkampf) verdoppelt. zudem erhältst du einen Vorteil. \textit{Bonus:} Waffe.

\subsection*{Schlachtruf} \label{sk:schlachtruf}
Mit einem mächtigen tiefen Gröllen stürmte sie in die Mitte des Kampfes. Ihre kläglich unterlegenden Gegner trauen sich kaum auch nur in ihre Richtung zu schauen. Ganz zu schweigen von einem Versuch eines Angriffes. \\
\textbf{Grundwert:} Willenskraft \\
\textbf{Kenntnisschwelle:} 3 \\
\textbf{Anforderungen:} Stationär $\Pik{}$ 16+ \\
\textbf{Effekt:}  Du kannst bis zur deiner nächsten Aktionsphase nicht von Kreaturen im Radius \textbf{FK} m als Ziel gewählt werden.

\subsection*{Überwältigen} \label{sk:ueberwaeltigen}
Wein ein wildes Tier stürmt der mächtige Rammbock von einem Ghul in die schwächelnde Verteidigung der überlaufenden Verteidiger. Sie habe keine Chance. Nicht mal auf Erbarmen. \\
\textbf{Grundwert:} Kraft \\
\textbf{Kenntnisschwelle:} 1 \\
\textbf{Anforderungen:} $\Herz{}$ 16+ \\
\textbf{Effekt:} \textit{Angriff (Physisch)}. Das Ziel erhält das Leiden \verweis{ef:erschoepft}.

% = = = = = = = = = = = = %

\section{Fähigkeiten der Krograg} \label{krogragskills}

\subsection*{Durchdringender Biss} \label{sk:durchdringender_biss}
Der massive Krograkkiefer kann im rechten Winkel angesetzt die Verteidigung eines Gegners knacken, als wäre sie gar nicht vorhanden.\\
\textbf{Grundwert:} Kraft \\
\textbf{Kenntnisschwelle:} 3 \\
\textbf{Anforderungen:} $\Kreuz{}$ 15+ \\
\textbf{Reichweite:} 0.5 m \\
\textbf{Effekt:} \textit{Angriff (Physisch)}. Bis zu \textbf{FK} Rüstung wird ignoriert.

\subsection*{Düsterer Schlag} \label{sk:duesterer_schlag}
Mit einer großen Rotation des Torsos schlägt der Krograg in einem aggressiven Tempo ihre scharfen Kralle ziel genau in ihren Gegenüber.\\
\textbf{Grundwert:} Geschicklichkeit \\
\textbf{Kenntnisschwelle:} 1 \\
\textbf{Maximale Kenntnis:} 15 \\
\textbf{Anforderungen:} Bewegend $\Pik{}$ 15+ \\
\textbf{Reichweite:} 0.5 m \\
\textbf{Effekt:} \textit{Angriff (Physisch)}. Kann nicht kritisch Verteidigt werden.

\subsection*{Eindringliche Gedanken} \label{sk:eindringlichegedanken}
\textbf{Grundwert:} Charisma \\
\textbf{Kenntnisschwelle:} 5 \\
\textbf{Anforderungen:} 18+ \\
\textbf{Effekt:} Der Effekt wird durch deinen gewälten Schöpfer entschieden. Solltest du bereits einen sekundären Schöpfer besitzen darfst du auch ihn wählen. In diesem Fall zählt jedoch dein FK nicht zur Probe.\\ \textbf{Primäre Schöpfer:}
\begin{itemize}
    \item \textbf{Varki:} Decke eine Karte auf. Sollte sie rot sein darfst du sie auf die Hand nehmen.
    \item \textbf{Keark:} Wähle eine Kreatur in 10 m. Sie erleidet 5 + FK Wunden.
    \item \textbf{Kearkii:} Lege eine Karten ab und ziehe daraufhin ebenso viele. Dies darfst du 1 + (FK/2) mal machen.
    \item \textbf{Moarln:} Wähle eine Kreatur in 5 m. Heile sie um die positive Differenz die dein Gesammtwert den Anforderungswert überschreitet.
\end{itemize}
\textbf{sekundäre Schöpfer:}
\begin{itemize}
    \item \textbf{Vand:} Wähle einen kreisförmigen Bereich in 10 m mit 1 + FK m Durchmesser. Kreaturen in diesem Bereich müssen falls möglich ihren Aktuellen Bewegungsstatus beibehalten und werden für die nächste Kampfphase als mit einer um 5 niedrigere Initiative behandelt.
    \item \textbf{Serkeij:} Wähle eine Kreatur in 2 + FK m Entfernung. Sie erhält die Vergiftung \textit{Gefrierendes Gift}.
    \item \textbf{Nierld:} Wähle zwei gleichgroße Kreaturen in 4 + FK m. Tausche ihre Positionen. Dies provoziert keine Reaktionsangriffe.
    \item \textbf{Aivir:} Gewähre allen befreundeten Kreaturen in 5 m ein Bonus von 1 + FK auf Willenskraft.
\end{itemize}

\subsection*{Rage} \label{sk:rage}
Ein wahrhaft wütender Krograg kann einen gewissen Punkt erreichen. Einen Punkt an dem es nichts mehr als pure Emotionen gibt.\\
\textbf{Grundwert:} Kraft \\
\textbf{Kenntnisschwelle:} 2 \\
\textbf{Anforderungen:} $\Herz{}$ 15+ \\
\textbf{Reichweite:} 0.5 m \\
\textbf{Effekt:} \textit{Angriff (Physisch)}.\textbf{Bonus:} Anzahl deiner erlittenen Wunden.

\subsection*{Sandsturm} \label{sk:sandsturm}
Ein monotones Summen frisst sich durch die zitternde Luft. Ein agieler, salziger Geschmack mischt sich unter den Duft der Beute und der Sand in der Luft raschelt liebevoll, als er mit der göttlichen Gabe des Krograg gesegnet wird.\\
\textbf{Grundwert:} Spirit \\
\textbf{Kenntnisschwelle:} 2 \\
\textbf{Anforderungen:} $\Herz{}$ \& $\Karo{}$ 20+ \\
\textbf{Effekt:} Erschaffe ein Sandsturm mit FK+2 m Radius um deine Position herum. Aktionen deren Sichtlinie zwischen Ziel und Agierender durch das Gebiet des Sturms führen dürfen diese nur ausführen, sollte nach Eintritt in dieses Gebiet die Distanz zum Ziel unter 0,5m liegt.

\subsection*{Schutz des Wissens} \label{sk:schutzdeswissens}
Eine beliebte Jagdtaktik des Krograg ist das überaschende Angreifen. Um einen Gegner derartig zu überrumpeln benötigt es eine gutes Zeitgefühl und einwenig Zeit zur Durchführung.\\
\textbf{Grundwert:} Intelligenz \\
\textbf{Kenntnisschwelle:} 3 \\
\textbf{Anforderungen:} $\Herz{}$ 20+ \\
\textbf{Reichweite:} 3 + FK m \\
\textbf{Effekt:} Ziehe zwei Karten. Lege eine Karte ab und lege eine Karte auf deinen Kartenstapel.

\subsection*{Spiritfluss} \label{sk:spiritfluss}
Ein jene göttliche Präsenz eins Krogrags kann sich in ihm in Form eine Feldes aus reinem Spirit manifestieren und durch in durch fließen.\\
\textbf{Effekt:} \textit{Passiv.} Wenn der göttliche Wille mit dir ist erhälst du +1 auf alle Eigenschaften.

\subsection*{Wasserbinden} \label{sk:wasserbinden}
Mit feinen Rillen in der schuppigen Haut kann der Krograg Wassersammeln in Kombination mit einem Körpersekret hat es eine heilende Wirkung.\\
\textbf{Grundwert:} Spirit \\
\textbf{Kenntnisschwelle:} 1 \\
\textbf{Maximale Kenntnis:} 20 \\
\textbf{Anforderungen:} Stationär $\Pik{}$ 15+ \\
\textbf{Effekt:} Du bindest Wasser aus der Luft, so das du deine Verteidigung bis zur deiner nächsten Aktionsphase verdoppelst und dich um FK*2 heilst. Du kannst dich die nächste Aktionsphase nicht Bewegen.



% = = = = = = = = = = = = %

\section{Fähigkeiten der Menschen} \label{menschenskills}

\subsection*{Beobachten} \label{sk:beobachten}
Die Beobachtung des Gegners kann entscheidene Vorteile im Kampf liefern und je nachdem sogar den Ausgang dominieren. 
\textbf{Effekt:} \textit{Passiv.} Wähle eine Kreatur in Reichweite. Solange du keine andere Aktiondurchführst deckt die gewählte Kreatur zur Verteidigung eine weitere Karte auf. Wähle eine Karte die sie nicht zum verteidigen nutzen darf.

\subsection*{Entschlossene Verteidigung} \label{entschlosseneverteidigung}
Der Wille des liebenden und ehrenden Menschens ist ein Schild, das nicht unterschätzt werden darf. Ein einziger zu entschlossener Krieger kann eine Horde an Gegner zu tiefst zu setzen wenn es um das Überleben seiner Nächsten geht.\\
\textbf{Grundwert:} Willenskraft \\
\textbf{Kenntnisschwelle:} 8 \\
\textbf{Anforderung:} Stationär \& 20+\\
\textbf{Effekt:} Wenn eine Kreatur bis zu deiner nächsten Aktionsphase in Bewegungsreichweite versucht in Nahkampfreichweite einer befreundeten Kreatur zu gelangen, darfst du dich zu ihr bewegen und einen Angriff mit diesem Gesammtwert durchführen (+ Waffenbonus). Nach dem Angriff darf die Kreatur sich entscheiden sich weiter zu bewegen und einen Reaktionsangriff zu ermöglichen oder im Nahkampf mit dir zu bleiben. Du erhälst + 1 Robustheit.

\subsection*{Freiheitsruf} \label{sk:freiheitsruf}
Aus irgendeinem Grund ist der Mensch sehr Freiheits vernarrt ohne so wirklich zu wissen was diese denn sei.\\
\textbf{Grundwert:} Charisma \\
\textbf{Kenntnisschwelle:} 4 \\
\textbf{Anforderung:} $\Pik{}$ 22+\\
\textbf{Effekt:} Alle befreundeten Kreaturen in 10 m Radius sind nicht mehr gebunden. Du und jede befreundete Kreatur dessen Bindung gelöst wurde erhalten einen Vorteil.


\subsection*{Geschultes Auge} \label{sk:geschultesauge}
Mit der Zeit Lernt man genau wo sich Verbrecher und Diebe befinden. Zudem bringen Plünder und Vergewaltigungstouren auch die Erfahrung des Opfersuchens mit sich.
\textbf{Grundwert:} Sinne \\
\textbf{Kenntnisschwelle:} 3 \\
\textbf{Reichweite:} 10 m \\
\textbf{Effekt:} Alle feindlichen Kreaturen in Reichweite deren Geschicktheitswert + 8 unter deinem Gesammtwert liegen verlieren ihre Tarnungseffekte.


\subsection*{Lebensraub} \label{sk:lebensraub}
Es ist unglaublich wie gut der Mensch in der Lage ist sich an Situationen anzupassen. Gerade in brenzligen Situation kommt es vor dass er sogar Spiritmanisfestionen nutzen kann um sich selber am Leben zu halten. \\
\textbf{Effekt:} \textit{Passiv.} Nach einem erfolgreichen Angriff mit einer Spiritaktion heile bis zu Element Tod Wunden, aber nicht mehr als du verursacht hast.
\textbf{Steigerung [15]:} +2 in Element Tod.
\textbf{Steigerung [25]:} +5 in Element Tod.

\subsection*{Schützenmeister} \label{sk:schützenmeister}
Jahre lange Erfahrung mit einer Waffe ist mit nichts auszugleichen. Wenn du genau weißt auf welche Distanz du triffst, wie man mit dem Wind spielt und wohin man zielen muss, wird man unglaublich gefährlich.\\
\textbf{Kenntnisschwelle:} 6 \\
\textbf{Effekt:} \textit{Pasiv}. Jede Kampfrunde darfst du 1 + FK mal bei einem Fernkampfangriff von deinem Gegner verlangen seine verteidgende Karte abzulegen und eine neue zu ziehen.

\subsection*{Schwere Lasten} \label{sk:schwere_lasten}
Die schwere Arbeit in Docks, auf Kränen oder in Werkstäten zwingt den Arbeiter dauern dazu Lasten zu tragen. Auf Dauer äußert sich dies häufig in Form von mächtigen, ausgeprägten Muskulaturen.\\
\textbf{Effekt:} \textit{Pasiv}. Erhöhe deine \textbf{Kraft} um eins. Der Arbeiter weiß wie man schwere Objekte in Bewegung versetzt. Du erhälst einen Boni von +2 auf alle Kraftproben in denen es darum geht Lasten zu bewegen.

\subsection*{Wachendes Auge} \label{sk:wachendes_auge}
Die Furcht die ein Mensch empfinden kann bringt auch einen entscheidenden Vorteil mit sich: er achtet sehr genau auf sein Umfeld und kann so seine Kameraden warnen. \\
\textbf{Grundwert:} Willenskraft \\
\textbf{Kenntnisschwelle:} 3 \\
\textbf{Anforderung:} Stationär 25+ \\
\textbf{Radius:}
\textbf{Effekt:} Unterbinde alle kritischen Effekte im Radius von dir bis zum Ende der nächsten Kampfrunde.

\subsection*{Wissen ist Macht} \label{sk:wissenistmacht}
Einen Gegner den man einmal angetroffen hat, wird man nicht so schnell wieder vergessen. Vor allem wenn davon dein Leben abhängen könnte.\\
\textbf{Grundwert:} Sinne \\
\textbf{Kenntnisschwelle:} 3 \\
\textbf{Anforderungen:} 17+ \\
\textbf{Reichweite:} 10 m \\
\textbf{Effekt:} \textit{Passiv} Nach dem Kampf gegen eine Kreatur kennst du nicht bloß die Fähigkeiten die du gesehen hast, sondern darfst dir vom Erzähler die restlichen Fähigkeiten sagen lassen.

% = = = = = = = = = = = = %

\section{Fähigkeiten der Morts} \label{mortsskills}

\subsection*{BBQ} \label{sk:bbq}
Der Legende nach entstand die Tradition des BBQ vor grob 54 Jahren in den Untiefen Norics, als einer Gruppe Ghule mit ihrem neuen MK-III Mrot Garblash die Nahrung ausgingen. Als sie nun mit Garblash jagen gingen entfesselte Garblash mit seinem Ultra Burna einen Flammenstrahl auf ein junges Kalb. Den sonst Rohfleisch gewohnten Ghule mundete die \textit{Babykuh} und so wurde das sprachlich modifizierbare BBQ. \\
\textbf{Grundwert:} Geschicklichkeit \\
\textbf{Kenntnisschwelle:} 2 \\
\textbf{Reichweite:} 6 m \\
\textbf{Effekt:} \textit{Angriff (Physisch)}. Kreaturen die sich in unmittelbarer Nähe des Zieles befinden, werden mit dem halben Gesammtwert ebenfalls getroffen. Verbraucht eine Ladung.

\subsection*{Blitzentladung} \label{sk:blitzentladung}
Ganz so als wäre der Himmel selber hinab gestiegen, entläd sich ein brennender Strahl Energie zu seinem nächsten Ziel um es schmerzhaft zu umarmen.\\
\textbf{Grundwert:} Spirit \\
\textbf{Kenntnisschwelle:} 1 \\
\textbf{Reichweite:} 2.5 m \\
\textbf{Effekt:} \textit{Angriff (Physisch)[X]}. Du triffst die Kreatur die dir am nächsten steht. Sollten geerdete metalische Gegenstände sich im Weg befinden so werden diese getroffen.

\subsection*{Flammeninferno} \label{sk:flammeninferno}
Wie ein Geysir der Flammen breitet sich eine glühende Decke aus verbrennenden Gases in der heißen Luft aus. Wenn der Schütze halbwegs gut zielt, kann er eigentlich gar nicht sein Ziel verfehlen.
\textbf{Grundwert:} Spirit \\
\textbf{Kenntnisschwelle:} 5 \\
\textbf{Reichweite:} Kegel mit Länge 4+FKm und Breite 1+FKm \\
\textbf{Anforderung:} Stationär \& $\Herz{}$ 18 +\\
\textbf{Effekt:} \textit{Angriff (Physisch)}. Jede Kreatur innerhalb des Kegels erleidet \textit{\nameref{ef:brennend}} (2+FK). Verbrauche 2 Ladungen.

\subsection*{Portal} \label{sk:portal}
Die Portaltechnologie ist eine äußerst umstrittene Angelegenheit. Das Problem mit den aufreißenden Löchern und den verschwindenen Körperteilen, ist für viele Orks keine Besorgnis, sondern viel eher ein Grund ihrem Mrot zugang zu Portaltechnologien zu verpassen. \\
\textbf{Grundwert:} Spirit \\
\textbf{Kenntnisschwelle:} 4 \\
\textbf{Maximale Kenntnis:} 3 \\
\textbf{Reichweite:} 2 m und 20 m \\
\textbf{Anforderung:} Stationär\\
\textbf{Effekt:} Wähle die Lokationen der Portale innerhalb der beiden Reichweiten. Verbrauche eine Ladung. Das Portal verbleibt für 1+FK Kampfrunden, bzw. bricht bereits ab sobald du nicht mehr stationär bist oder das Portal schließen möchtest. Das schließen ist auch außerhalb deiner Aktionsphase möglich. Für jede biologische Kreatur (keine \textit{mechanischen}) die sich durch das Portal bewegt, deckt der Mrot ein Karte von seinem Stapel auf. Der Effekt auf die Kreatur wird durch den Karteneffekt bestimmt:
\begin{itemize}
    \item \textbf{Zahlenwert:} Die glückliche Kreatur überlebt unbeschadet die Reise.
    \item \textbf{Bube:} Das Portal ist ein wenig fehler haft in der Zeitsynchronisierung. Die Kreatur taucht erst in seiner nächsten Bewegung, wieder aus dem anderen Portal auf. Sollte das Portal in der Zwischenzeit geschlossen werden bleibt die Kreatur so lange in dem Portal gefangen bis der Portalerschöpfer ein neues öffnet.\\
    \item \textbf{Dame:} Mit einer starken Entladung bricht das Portal überlasstet zusammen und verleiht nach dem Benutzen des Portals der Kreatur den Effekt \textit{\nameref{ef:geschockt}} (2+FK).
    \item \textbf{König:} Schneidene Risse durchbeißen das Portal und schneiden tief in die Psyche des Benutzers. Er erleidet die Hälfte seiner maximalen Psychewunden.
    \item \textbf{As:} So sollte das eigentlich nicht funktionieren oder? Das Portal skaliert die Größe der Kreatur falsch und verändert so die physischen Fähigkeiten der Kreatur. Sie darf eines ihrer Körperattribute um eins erhöhen.
\end{itemize}

\subsection*{Rücksichtsloses Schlagen} \label{sk:rücksichtsloses_schlagen}
.......................\\
\textbf{Grundwert:} Kraft \\
\textbf{Kenntnisschwelle:} 3 \\
\textbf{Maximale Kenntnis:} 7 \\
\textbf{Anforderung:} $\Pik{}$\\
\textbf{Reichweite:} 0.5 m\\
\textbf{Effekt:} \textit{Angriff (Physisch)}. Darf kostenfrei nach einer Schlagenaktion verkettet werden.

\subsection*{Scannen} \label{sk:scannen}
Mit einem modernen \nameref{entw:high-tex} kann der Mrot seine Umgebung und vor allem seinen Gegner genauer Betrachten und seine Schwachstellen erkennen.\\
\textbf{Grundwert:} Sinne \\
\textbf{Anforderung:} 13+ \\
\textbf{Effekt:} Du erhälst einen Vorteil.

\subsection*{Schockladung} \label{sk:schockladung}
Mit einem mächtigen Surren zündet der Mrot einen zusätzlichen Energieschub und erzeugt ein mächtiges Blitzfeld um sich herum, um seine ungedeckt stehen zu lassen.\\
\textbf{Grundwert:} Spirit \\
\textbf{Kenntnisschwelle:} 4 \\
\textbf{Maximale Kenntnis:} 7 \\
\textbf{Reichweite:} Kreis mit Radius 2+FK m\\
\textbf{Anforderung:} Stationär\\
\textbf{Effekt:} \textit{Angriff (Physisch)}. Alle Kreaturen innerhalb des Kreises um dich herum werden getroffen. Kreaturen die nicht ausreichend verteidigen können erleiden keine Wundenverluste sondern erhalt den Effekt \textit{\nameref{ef:geschockt}} (1).

\subsection*{Tarnen} \label{sk:tarnen}
Die verschwimmenden Konturen des Mrots erschweren es dem Feind ihre Lokation ein zu schätzen so dass der Mort ihre ungenauen Angriffe besser überstehen dürfte.\\
\textbf{Grundwert:} Spirit \\
\textbf{Kenntnisschwelle:} 5 \\
\textbf{Maximale Kenntnis:} 6 \\
\textbf{Anforderung:} 16 +\\
\textbf{Effekt:} Du tarnst dich und erhälst somit einen Bonus von 5 auf deine Verteidigung. Sobald du Wunden erleidest oder selber angreifst verblasst die Tarnung. Erhalte ein Vorteil.

\subsection*{Übertaktung} \label{sk:uebertaktung}
Morts können durch geschickte Hände auch noch effizienter gemacht werden. Dabei wir durch das übertakten aus ihren Kernen so viel wie nur möglich rausgehöhlt.\\
\textbf{Effekt:} \textit{Passiv.} Du benutzt den Kern dauerhaft überteuert also mit +5 Energie, bekommst dennoch keinen Malus.
% = = = = = = = = = = = = %

\section{Fähigkeiten der Skriva} \label{skrivaskills}

\subsection*{Eingraben} \label{sk:eingraben}
Skriva graben gigantische Tunnelsysteme in denen Kolonien von Skriva hausen. Damit ist es für einen einzelnen Skriva auch nicht schwierig sich einzugraben. \\
\textbf{Grundwert:} Geschicklichkeit \\
\textbf{Kenntnisschwelle:} 1 \\
\textbf{Maximale Kenntnis:} 15 \\
\textbf{Effekt:} Wenn du eingegraben bist regenerierst du \textbf{FK} Wunden pro Aktionsphase. Du kannst nur die Aktion Vorausplannen nutzen.

\subsection*{Furie} \label{sk:furie}
Ein wahrhaft wütender Skriva ist unglaublich brutal und rücksichtslos. Er lebt im hier und jetzt ohne einen Gedanken an sein eigenes Leben oder gar eines seiner Gefährten zu verschwenden. Sein Ziel ist seine einzige Priorität. \\
\textbf{Grundwert:} Kraft \\
\textbf{Kenntnisschwelle:} 5 \\
\textbf{Anforderung:} 25+ \\
\textbf{Reichweite:} 0.5 m \\
\textbf{Effekt:} \textit{Angriff (Physisch)}. Du darfst bis zu \textbf{FK} Karten für diese Aktion nutzen.

\subsection*{Starkes Gewebe} \label{sk:starkes_gewebe}
Eine schwere Kindheit in dem Schatten der Verachtung fordert viel von dem Körper eines Skrivas. Um ein solches Leben führen zu können bilden sich bereits früh verstärkte Gewebe die Schläge und Stütze abhalten können. \\
\textbf{Effekt:} \textit{Passiv.} Dein Wundenmaximum steigt um die Hälfte deiner Robustheit (abgerundet).

\subsection*{Schattenläufer} \label{sk:schattenläufer}
Die Skriva sind seid Generationen schon Kreaturen der Schatten. Auch wenn sie nicht unter offenen Bereichen leiden fühlen sie sich schnell ungedeckt und unwohl. Im Gegenzug sind sie jedoch hervoragende Jäger in den Schatten und unglaublich schwer auszumachen, wenn sie nicht gesehen werden wollen.
\textbf{Grundwert:} Gewandheit \\
\textbf{Kenntnisschwelle:} 6 \\
\textbf{Maximale Kenntnis:} 5 \\
\textbf{Anforderung:} 18 +\\
\textbf{Effekt:} Du musst dich innerhalb von einem Meter Entfernung von einem Objekt befinden das min. eine Größenortnung größer als du ist. Du tarnst dich und erhälst somit einen Bonus von 8 auf deine Verteidigung. Sobald du Wunden erleidest, eine Aktion außerhalb eines Schatten bzw. im Nahkampf mit einer Kreatur beendest oder selber angreifst, verblasst die Tarnung. Erhalte ein Vorteil.

\subsection*{Totstellen} \label{sk:totstellen}
Um sich gegen mächtige Feide zu entkommen, haben manche Skriva gelernt sich tot zu stellen. \\
\textbf{Grundwert:} Willenskraft \\
\textbf{Kenntnisschwelle:} 4 \\
\textbf{Anforderung:} 20+ \\
\textbf{Effekt:} Du kannst nicht aktiv von Gegnern als Ziel gewählt werden. Außer sie schaffen eine schwere Intelligenzprobe. Sobald du eine Aktion ausführst die nicht Blocken oder Vorausplanen ist endet der Effekt. Du gilst als Leiche.

\subsection*{Toxinwolke} \label{sk:toxinwolke}
Das närrische Funkeln der Vorfreude in den Augen eines Skrivas bedeutet seltens etwas gutes. Hört man Sekunden später ein sanftes Zischens, richt schwefelartige Gerüche und sieht ein ungleichmäßig verteiltes Grün aufsteigen, so weiß man dass man definitiv nicht atmen sollte! \\
\textbf{Grundwert:} Spirit \\
\textbf{Kenntnisschwelle:} 4 \\
\textbf{Maximale Kenntnis:} 5 \\
\textbf{Anforderung:} Stationär \& $\Karo{}$ 18 +\\
\textbf{Radius:} 1 m \\
\textbf{Effekt:} Alle Kreaturen innerhalb des Wirkungsradius müssen eine Robustheitsprobe (17+) bestehen oder werden Opfer deines aktiven Gifts.
\textbf{Bonus:} Toxin.

% = = = = = = = = = = = = %

\section{Fähigkeiten der Wareguards} \label{wareguardskills}

\subsection*{Allesfresser} \label{sk:allesfresser}
Mit einer flinken Bewegung sticht der wendige Skriva seinen Klauenbesetzen Schwanz in die Organe seines Gegners. Das übertragende Gift kann nun frei im langsam zerfallenden Körper propagieren.\\
\textbf{Reichweite:} 0.5 m \\
\textbf{Effekt:} Wähle eine Leiche in Reichweite und beginn sie aufzufressen. Decke eine Karte auf. Solltest du eine Zahl aufdecken, regeneriest du eine Wert an Wunde der dem Zahlenwert entspricht. Bube, Dame und König stellen 5 Gift- und 5 Psychewunden wiederher. Solltest du ein As aufdecken darfst du eine schwere Wunde die du in \textit{diesem} Kampf erlitten hast streichen.

\subsection*{Aufsteigen} \label{sk:aufsteigen}
Der Wareguard ist ein Tier der Lüfte. Er liebt das Gleiten im Wind und das schnelle majestätische Gleiten hinab zu seinem Ziel.
\textbf{Grundwert:} Gewandtheit \\
\textbf{Kenntnisschwelle:} 5 \\
\textbf{Anforderung:} 16+ \\
\textbf{Effekt:} Du fliegst. Dein Fliegenwert erhöht sich um bis zu deinem Kenntniswert +2.

\subsection*{Ausweiden} \label{sk:ausweiden}
Wareguard lieben das Gefühl in die straffe Haut ihrer panischen Beute zu beißen. Dieser euphorische Schub motiviert sie zu den brutalsten Verstümmelungen.
\textbf{Grundwert:} Geschicklichkeit \\
\textbf{Kenntnisschwelle:} 2 \\
\textbf{Anforderung:} $\Herz{}$ 19+ \\
\textbf{Reichweite:} 0.5 m \\
\textbf{Effekt:} \textit{Angiff (Physisch)}. War das Ziel zuvor unverwundet verdoppel die zugefügte Wunden des Angriffes.

\subsection*{Einstürmen} \label{sk:einstürmen}
Wenn der massive Körper eines Wareguards einmal in Bewegung ist, dann wird es schwer werden ihn wieder aufzuhalten. Sein Gewicht reicht locker aus um Holzwände und Schädel einzustürmen.\\
\textbf{Grundwert:} Kraft \\
\textbf{Kenntnisschwelle:} 6 \\
\textbf{Anforderung:} Bewegt \\
\textbf{Effekt:} Wähle einen geraden Pfad von 4 m Länge den du durchstürmen möchtest. Jede Kreatur die innerhalb von einem halben Meter dieses Pfades liegt muss eine vegleichende Initiativprobe bestehen oder wird Opfer eines \textit{Schlagenangriffs} mit dem Kartenwert dieser Aktion. Alle Kreaturen werden orthogonal zur Laufliene aus dem Bereich geschoben. Solltest du durch Gelände stürmen, muss der Spielleiter entscheiden ob dieses nachgibt.

\subsection*{Entziehen} \label{sk:entziehen}
Mit einem überraschenden Flügelschlag und einer Menge aufsteigenden Drecks lenkt der Wareguard seine Gegner ab und kann sich somit gefahrlos von seinem Gegenüber lösen.\\
\textbf{Effekt:} Du kannst dich einmal pro Kampfrunde frei aus einem Nahkampf zurückziehen.
\textbf{Steigerung [15]:} Du kannst dich nun zweimal pro Kampfrunde frei aus einem Nahkampf zurückziehen.
\textbf{Steigerung [25]:} Du kannst dich frei aus dem Nahkampf zurückziehen.

\subsection*{Erheben} \label{sk:erheben}
Viele Vögel und einige andere Lebenformen sind in der Lage ihre Flügel zu nutzen um sich in die Lüfte zu erheben und sich so eine Vorteil über andere Lebensformen zu sichern.\\
\textbf{Effekt:} Du \textit{fliegst}. Erhöhe deinen Fliegenwert um bis zu deinem Kenntniswert + 4.

\subsection*{Verschleppen} \label{sk:verschleppen}
Aus den Zeiten der Alten haben die Wareguards noch immer den Drang mächtige Gegner zu ihren Meisterinen zu bringen. Auch wenn dies heute keinen Nutzen mehr erbringt, so kann ihre Eigenart des Verschleppen noch immer im Kampf sinnvoll sein.\\
\textbf{Grundwert:} Kraft \\
\textbf{Vergleichsprobe (Ziel):} Kraft (ST) oder Gewandheit (BW)
\textbf{Kenntnisschwelle:} 4 \\
\textbf{Reichweite:} 1 m \\
\textbf{Effekt:} Wähle ein Ziel in Reichweite. Sollte das Ziel die Vergleichsprobe nicht dominieren so darfst du es während deiner Bewegung mit zerren. Platziere es am Ende deiner Bewegung innerhalb der Reichweite an der Stelle die am nächsten zum Ausgangpunkt ist.