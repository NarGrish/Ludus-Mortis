\chapter{Andere Kreaturen}

\tcbset{fonttitle=\bfseries\large,fontupper=\normalsize\sffamily,colback=myred,colframe=black,colbacktitle=gray!50!white,coltitle=black, title}

Die Heimlichen Lande sind voller verschiedensten Lebensformen und Geschichten. Eine jede Spezies ist einzigartig und bringt ihre eigenen Besonderheiten mit sich.


\section*{Daera}

\begin{tcolorbox}[title= Charakteristiken,colbacktitle=mysilver, tabulars={@{\extracolsep{\fill}\hspace{1mm}}ll@{\hspace{1mm}}}, boxrule=0.5pt]
    \textbf{Gefahrenstufe:} & 2 \\
    \textbf{Erfahrungspunkte:} & 30 EP \\
    \textbf{Rasse:} & Mechanismus \\
    \textbf{Wunden:} & 26 \\
    \textbf{Blocken / Ausweichen:} & 10 / 12 \\
    \textbf{Gift / Resistenz:} & 39 / 9 \\
    \textbf{Psyche / Widerstand:} & 17 / 15 \\
    \textbf{Bewegung:} & 5 \\
    \textbf{Aktionspunkte:} & 5
\end{tcolorbox}

\begin{tcolorbox}[title= Eigenschaften, colbacktitle=mysilver, tabulars={@{\extracolsep{\fill}\hspace{1mm}}cccccc@{\hspace{1mm}}}, boxrule=0.5pt]
    \textbf{Rob} & \textbf{Gew} & \textbf{Spi} & \textbf{Ini}  & \textbf{Kra} & \textbf{Ges} \\
    10 & 12 & 11 & 10 & 11 & 17\\ \hline
    \textbf{Int} & \textbf{Cha} & \textbf{Emo} & \textbf{Krea}  & \textbf{Sin} & \textbf{Wil} \\
    11 & 17 & 7 & 14 & 4 & 15
\end{tcolorbox}

Die Daeren wurde von dem Daemon Klaerlnich und seinem ghulischen Freund Graklas erschaffen. Sie sollten zu Beginn lediglich den Lüsten der beiden genügen und sich gegenseitig in der Arena brutal auseinander nehmen. Als die Daeren jedoch effektiver wurden entschieden ihre Meister sie zu nutzen um ihre Gegner auszuschalten. Diese metallgegossenen Frauen sind sehr flink und effektiv. Seid ihrer Erschaffung 143 m.Z. stellten die beiden Erschaffer bis zu 250 Daera her. Nach ihrer Ermordung wurden ihre Metallfrauen 213 m.Z. versteigert und befinden sich heutzutage in den Händen verschiedenster mächtiger Personen.

\subsection*{Niederschlagen}
Wie eine wilde Bestie schlägt das graumetallisch glänzende Wesen an auf ihn ein zuschlagen. Immer und immer wieder. \\
\textbf{Effekt:} \textit{Passiv}. Daeras zählen mit zwei freien Händen als mit zwei Waffen ausgerüstet.

\epigraph{Sieh wie tanzen! Sieh wie sie schreien! Sie wie sie einander zerschlagen. Das ist Kunst, das ist die Zukunft mein Freund. Wir werden sie perfektionieren, zerschlagen und wieder aufbauen!}{\textit{Klaerlnich, daemonischer Psychopath}}

% _ _ _ _ _ _ _ _ _ _ _ _ _ _ _ _ %

\section*{Kriegsläufer} \label{NPC:Kriegsläufer}

\begin{tcolorbox}[title= Charakteristiken,colbacktitle=mysilver, tabulars={@{\extracolsep{\fill}\hspace{1mm}}ll@{\hspace{1mm}}}, boxrule=0.5pt]
    \textbf{Gefahrenstufe:} & 18 \\
    \textbf{Erfahrungspunkte:} & 2500 EP \\
    \textbf{Rasse:} & Mechanismus \\
    \textbf{Größe:} & gigantisch \\
    \textbf{Wunden:} & 200 \\
    \textbf{Blocken / Ausweichen:} & 19 / 5 \\
    \textbf{Gift / Resistenz:} & 41 / 19 \\
    \textbf{Psyche / Widerstand:} & 31 / 15 \\
    \textbf{Bewegung:} & 4 \\
    \textbf{Aktionspunkte:} & 5
\end{tcolorbox}

\begin{tcolorbox}[title= Eigenschaften, colbacktitle=mysilver, tabulars={@{\extracolsep{\fill}\hspace{1mm}}cccccc@{\hspace{1mm}}}, boxrule=0.5pt]
    \textbf{Rob} & \textbf{Gew} & \textbf{Spi} & \textbf{Ini}  & \textbf{Kra} & \textbf{Ges} \\
    19 & 5 & 16 & 4 & 18 & 13\\ \hline
    \textbf{Int} & \textbf{Cha} & \textbf{Emo} & \textbf{Krea}  & \textbf{Sin} & \textbf{Wil} \\
    9 & 1 & 4 & 7 & 13 & 15
\end{tcolorbox}

Der Begriff des Mrot ist zwar unglaublich dehnbar und doch hört er bei einer gewissen Größe der Konstruktion auf. Die riesigen Kriegsläufer der Nordens wurden von den Gauschtig-Orks in den Zeiten der Nordkriegen im Westen des heutigen Valreiks produziert. Ihre Herstellung wurde nach dem Friedensschluss eingestellt und nach der Einsetzung der gemischten Valarmee wurden sie als unmittelbare Gefahren des Friedens eingeschränkt. Nun findet man bloß noch vereinzelte Kriegsläufer in den Schneeebenen und in den weißen Landen. Ihre Häuser großen Körper werden von Ultracores betrieben und sind mit mächtige Belagerungswaffen ausgestattet. Auch wenn ihr Metall nach all den Jahren mit Rost befleckt wurde sind ihre mächtigen Stahlschienen noch immer ein effektiver Schutz gegen Angreifer mit natürlichen Klauen und Zähnen.

\subsection*{Schreiten}
Mit einem ununterbrochene Stapfen rückt der riesige Koloss stetig vor. Trotz des massiven Feuerns hält er seine Bewegung aufrecht und lässt sich nicht beirren. \\
\textbf{Effekt:} \textit{Passiv}. Der Kriegsläufer zählt nach seiner Bewegung als Stationär, außer ein externer Effekt behauptet etwas anderes.

\subsection*{Schwere Atelier}
\textbf{Grundwert:} 13 (Ges) + 4 (Schweres Geschütz) \\
\textbf{Reichweite:} 140 m \\
\textbf{Radius:} 2 m \\
\textbf{Effekt:} \textit{Angriff (Physisch)}. Decke eine weiter Karte auf und lege die kleinste ab. \textit{Schwere Atelier} kann nur jede zweite Runde verwendet werden.  \\

\epigraph{Ein Kreischen klang nieder von den mächtigen Bäumen die sie durch den Schneesturm sahen. Viel zu regelmäßig für einen natürlichen Ursprung. Wind sollte anders klingen. Und Bäume sollten still stehen.}{\textit{Marcberg Seissel, minotaurischer Wegführer}}

% _ _ _ _ _ _ _ _ _ _ _ _ _ _ _ _ %

\section*{Leichenfresser} \label{NPC:Leichenfresser}

\begin{tcolorbox}[title= Charakteristiken,colbacktitle=myviolet, tabulars={@{\extracolsep{\fill}\hspace{1mm}}ll@{\hspace{1mm}}}, boxrule=0.5pt]
    \textbf{Gefahrenstufe:} & 1 \\
    \textbf{Erfahrungspunkte:} & 10 EP \\
    \textbf{Rasse:} & Wiedergänger \\
    \textbf{Größe:} & mittel \\
    \textbf{Wunden:} & 25 \\
    \textbf{Blocken / Ausweichen:} & 12 / 11 \\
    \textbf{Gift / Resistenz:} & 10 / 11 \\
    \textbf{Psyche / Widerstand:} & 13 / 7 \\
    \textbf{Bewegung:} & 4 m \\
    \textbf{Aktionspunkte:} & 4
\end{tcolorbox}

\begin{tcolorbox}[title= Eigenschaften, colbacktitle=myviolet, tabulars={@{\extracolsep{\fill}\hspace{1mm}}cccccc@{\hspace{1mm}}}, boxrule=0.5pt]
    \textbf{Rob} & \textbf{Gew} & \textbf{Spi} & \textbf{Ini}  & \textbf{Kra} & \textbf{Ges} \\
    12 & 11 & 3 & 7 & 11 & 8\\ \hline
    \textbf{Int} & \textbf{Cha} & \textbf{Emo} & \textbf{Krea}  & \textbf{Sin} & \textbf{Wil} \\
    7 & 3 & 6 & 9 & 13 & 7
\end{tcolorbox}
Die Leichenfresser sind Kreaturen der Dunkelheit. Ihre blasse Haut überspannt den knochigen trägen Körper wie ein feuchtes Laken. Die missgestellten Glieder weisen noch nach Generationen von Auferstehungsiterationen die Ähnlichkeit zu ihren vorherigen Leben als Ghule im Dienste des Berstimperiums auf. Der Fluch für den so viele fehlgeleitete Kultisten des letzten Imperators ihre Seele opferten, zwingen diese arme Diener der Nation, Tag um Tag neu aufzustehen. Ihre Innereien regenerieren Nacht um Nacht im zaudern des Spiritfeldes, bis sie soweit zerfallen sind, dass selbst der Geist der Fanatiker nicht mehr erkennen kann welche Stücke zueinander gehören.\\
Diese Seelenlosen Kreaturen, kommen jede Nacht aus dem Schatten der Erde hervor um nach Leichen zum Fraß zu suchen. Denn sollten sie keine über wochen hinweg finden, würde das selbst der alte Fluch nicht in der Lage sein die materiellen Ressourcen zur Wiederherstellung zur Verfügung zu stellen. Die verzweifelten Leichenfresser gehen teilweise sogar soweit und versuchen sich daran ihr Fressen selber zu erlegen. Auch wenn dies nahezu nie Erfolg bringen wird, sehen diese Wesen keinen besseren Weg ihrem Auftrag folgen zu können. Denn wie sonst sollte das Erbe des Imperator Krailgaa erhalten bleiben?

\subsection*{Ausnehmen}
Um an physisch Ressourcen zur Regeneration zu haben benötigt der Leichenfresser Fleisch zwischen seinen Zähnen. Ob er dabei das Fleisch von Lebenden oder Toten mit seinen Reißern hervorholt spielt keine Rolle. \\
\textbf{Grundwert:} 11 (Kra) \\
\textbf{Anforderung:} Rot \\
\textbf{Reichweite:} 0.5 m \\
\textbf{Effekt:} \textit{Angriff(Physisch)}.  Heile dich um die verursachten Wunden. Dieser Angriff kann auch gegen Leichen ausgeführt werden. Hierbei würde die Anforderung \textit{Rot} gegen \textit{Stationär} getauscht werden.

\subsection*{Giftbeißer}
Die ständige Erneuerung durch den Fluch bring die Flüssigkeiten des Körpers in ganz neue Verwesungszustände. Somit wird unteranderem der Speichel zu einer sauren Mischung aus zersetzenden Substanzen, die das Verzehren von alten Fleisch vereinfachen. \\
\textbf{Effekt:} Wenn du einer Kreatur Wunden zufügst füge Ihr genauso viele Giftwunden zu.

\epigraph{Steh auf Heer des Imperiums! Richtet eure Rücken gen Himmel, und maschiert in die Zukunft! Unser Traum wärt ewig, bis die Zeit selbst vergilbt und die Erde zerbricht!}{\textit{Nic'Soul, Nekromant des letzten Imperators}}

% _ _ _ _ _ _ _ _ _ _ _ _ _ _ _ _ %

\section*{Mortak}

\begin{tcolorbox}[title= Charakteristiken,colbacktitle=mybrown, tabulars={@{\extracolsep{\fill}\hspace{1mm}}ll@{\hspace{1mm}}}, boxrule=0.5pt]
    \textbf{Gefahrenstufe:} & 5 \\
    \textbf{Erfahrungspunkte:} &  150 EP \\
    \textbf{Rasse:} & Kratzer \\
    \textbf{Größe:} & groß \\
    \textbf{Merkmale:} & \textit{\nameref{ef:schwimmer}}\\
    \textbf{Wunden:}  61 \\
    \textbf{Blocken / Ausweichen:} & 16 / 10 \\
    \textbf{Gift / Resistenz:} & 29 / 16 \\
    \textbf{Psyche / Widerstand:} & 15 / 8 \\
    \textbf{Bewegung:} & 5 \\
    \textbf{Aktionspunkte:} & 5\\
    \textbf{NK-Waffe (Klauen)}: & +2, 0.5 m
\end{tcolorbox}

\begin{tcolorbox}[title= Eigenschaften, colbacktitle=mybrown, tabulars={@{\extracolsep{\fill}\hspace{1mm}}cccccc@{\hspace{1mm}}}, boxrule=0.5pt]
    \textbf{Rob} & \textbf{Gew} & \textbf{Spi} & \textbf{Ini} & \textbf{Kra} & \textbf{Ges} \\
    16 & 10 & 3 & 6 & 18 & 7\\ \hline
    \textbf{Int} & \textbf{Cha} & \textbf{Emo} & \textbf{Krea} & \textbf{Sin} & \textbf{Wil} \\
    3 & 3 & 5 & 7 & 9 & 8
\end{tcolorbox}

In allerlei sumpfigen und feuchten Gebieten leben still und heimlich die Mortak. Die wenige Zeit die der Mortak nicht mit einem matschigen Schlaf verbringt, schleicht oder lauert er im Grünen um unvorsichtige Beute zu überraschen. Der grüne gepanzerte Rücken des Mortak eignet sich hervorragend um mit der Umgebung zu verschmelzen und selbst auf wenige Meter Abstand nicht erkannt zu werden. Sobald ihr Opfer unmittelbar vor ihnen ist überläuft der Mortak sein Opfer förmlich und reißt wenn nötig Innereien aus seinem Opfer. Sobald es tot und erlegt ist ziehen sie die Leiche zu einer abgelegenen Stelle und fressen fröhlich die Aufgeweichten Innereien. Den Restliche Körper lassen sie entweder liegen oder lassen ihn im Unterholz verschwinden.
Als sehr einsame Tiere kommt es selten dazu dass große Anzahlen an Mortaks sich in der selber Umgebung befinden. Gerade männliche Tiere fliehen tendenziell von Artgenossen, da die vergewaltigenden Weibchen ihren Geschlechtspartner umbringen um die Kinder in seinem Bauch aufwachsen und durch seine Inneren Körperteile füttern zu lassen.

%  = = = = = = = = =  %
\section*{Plünderer (Ghul)}

\begin{tcolorbox}[title= Charakteristiken, colbacktitle=myviolet, tabulars={@{\extracolsep{\fill}\hspace{1mm}}ll@{\hspace{1mm}}}, boxrule=0.5pt]
    \textbf{Gefahrenstufe:} & 1 \\
    \textbf{Erfahrungspunkte:} & 10 EP \\
    \textbf{Rasse:} & Mischling \\
    \textbf{Größe:} & mittel \\
    \textbf{Wunden:} & 40 \\
    \textbf{Schlagen ST/BW:} & 11 (+2) / 5 (+2) \\
    \textbf{Blocken / Ausweichen:} & 10 (+1) / 6 (+1)\\
    \textbf{Gift / Resistenz:} & 12 / 10 \\
    \textbf{Psyche / Widerstand:} & 11 / 6 \\
    \textbf{Bewegung:} & 5 \\
    \textbf{Aktionspunkte:} & 5 \\
    \textbf{NK-Waffe (Axt):} & +2, 1.5 m \\
    \textbf{Rüstung (Leder)}: & +1 Rüstung
\end{tcolorbox}

\begin{tcolorbox}[title= Eigenschaften, colbacktitle=myviolet, tabulars={@{\extracolsep{\fill}\hspace{1mm}}cccccc@{\hspace{1mm}}}, boxrule=0.5pt]
    \textbf{Rob} & \textbf{Gew} & \textbf{Spi} & \textbf{Ini}  & \textbf{Kra} & \textbf{Ges} \\
    10 & 6 & 6 & 8 & 11 & 5\\ \hline
    \textbf{Int} & \textbf{Cha} & \textbf{Emo} & \textbf{Krea}  & \textbf{Sin} & \textbf{Wil} \\
    4 & 6 & 7 & 10 & 6 & 6
\end{tcolorbox}

\subsection*{Schmetterschlag}
Mit einer mächtigen Bewegung nimmt der Ghul jegliche ihm zur Verfügung stehende Kraft zusammen und zwing seine rohe Waffe nieder gen Boden. Einem Regen aus Stahl, Blut und Stückchen zerschellt sie an dem triefende Aufprall am Gegner. \\
\textbf{Grundwert:} Kraft \\
\textbf{Anforderungen:} $\Kreuz{}$ \\
\textbf{Reichweite:} 0.5 m \\
\textbf{Effekt:}  Verbrauche einen weiteren Aktionspunkt. \textit{Angriff (Physisch)}. Ignoriere die Rüstungs/Schild Bonis der Kreatur. Falls eine Waffe für diese Aktion verwendet wird Zerbricht diese (nur Nahkampfwaffen können genutzt werden) falls die oberste Karte ein Kreuz ist richtet dabei Wunden in Höhe des Bonuses an.

\subsection*{Wutanfall}
Emotionen spielen eine zentralle Rolle im Leben der Ghule. Sie lieben das Gefühl von Dominanz, sie hassen die Hilflosigkeit und sie können echt wütend werden. Eine solch wütende Kreatur in den Feindreihen zu sehen, bringt einen schnell dazu seine Pläne um zu ändern. \\
\textbf{Grundwert:} Emotionen \\
\textbf{Anforderungen:} $\Herz{}$ 16+ \\
\textbf{Effekt:} \textit{Angriff (Psyche)}: Falls die Kreatur diese Aktionsrunde noch nicht dran war, kannst du dich entscheiden keinen Schaden zu machen, dafür muss die Kreatur in seiner nächsten Aktionsphase eine Vorausplannen Aktion durchführen. 
% _ _ _ _ _ _ _ _ _ _ _ _ _ _ _ _ %

%  = = = = = = = = =  %
\section*{Plünderer (Mensch)}

\begin{tcolorbox}[title= Charakteristiken, colbacktitle=myskin, tabulars={@{\extracolsep{\fill}\hspace{1mm}}ll@{\hspace{1mm}}}, boxrule=0.5pt]
    \textbf{Gefahrenstufe:} & 1 \\
    \textbf{Erfahrungspunkte:} & 10 EP \\
    \textbf{Rasse:} & Mischling \\
    \textbf{Größe:} & mittel \\
    \textbf{Wunden:} & 37 \\
    \textbf{Schlagen ST/BW:} & 8 (+1) / 8 (+1) \\
    \textbf{Blocken / Ausweichen:} & 8 (+1) / 8 (+1) \\
    \textbf{Gift / Resistenz:} & 10 / 8 \\
    \textbf{Psyche / Widerstand:} & 10 / 8 \\
    \textbf{Bewegung:} & 5 \\
    \textbf{Aktionspunkte:} & 5 \\
    \textbf{NK-Waffe (Schwert):} & +1, 1 m \\
    \textbf{NK-Fernkampf (Kurzbogen)}: & +1, 20 m \\
    \textbf{Rüstung (Leder)}: & +1 Rüstung
\end{tcolorbox}

\begin{tcolorbox}[title= Eigenschaften, colbacktitle=myskin, tabulars={@{\extracolsep{\fill}\hspace{1mm}}cccccc@{\hspace{1mm}}}, boxrule=0.5pt]
    \textbf{Rob} & \textbf{Gew} & \textbf{Spi} & \textbf{Ini}  & \textbf{Kra} & \textbf{Ges} \\
    8 & 8 & 5 & 10 & 8 & 8\\ \hline
    \textbf{Int} & \textbf{Cha} & \textbf{Emo} & \textbf{Krea}  & \textbf{Sin} & \textbf{Wil} \\
    5 & 8 & 6 & 9 & 4 & 8
\end{tcolorbox}

\subsection*{Koordinierter Angriff}
Mit schnellen, präzisen Angriffen kann selbst ein übermächtiger Feind in die Knie gezwungen werden.\\
\textbf{Grundwert:} Intelligenz \\
\textbf{Anforderung:} $\Karo{}$ 15+\\
\textbf{Reichweite:} 10 m \\
\textbf{Effekt:} Braucht an stelle von 1em 3 Aktionspunkte. Bestimme ein Ziel jede befreundete Kreatur in Reichweite zur gewählten Kreatur (du auch) darf eine Schlagen (Werfen oder Schießen) Probe durchführen.

\subsection*{Taktieren}
Mit seinen vorsichtigen Vorgehensweisen ist der Mensch häufig zurückhaltend und schätzt sein Leben über dem Tod seines Feindes.\\
\textbf{Grundwert:} Intelligenz \\
\textbf{Anforderung:} Stationär $\Kreuz{}$ 15+ \\
\textbf{Effekt:} Befreundete Kreaturen im 10 m Radius haben die nächste Aktionsphase kostenlos Verketten bei ihren Aktionen.
% _ _ _ _ _ _ _ _ _ _ _ _ _ _ _ _ %

\section*{Rothorn}

\begin{tcolorbox}[title= Charakteristiken,colbacktitle=mybrown, tabulars={@{\extracolsep{\fill}\hspace{1mm}}ll@{\hspace{1mm}}}, boxrule=0.5pt]
    \textbf{Gefahrenstufe:} & 3 \\
    \textbf{Erfahrungspunkte:} & 55 EP \\
    \textbf{Rasse:} & Kratzer \\
    \textbf{Größe:} & groß \\
    \textbf{Wunden:} & 43 \\
    \textbf{Blocken / Ausweichen:} & 9 / 14 \\
    \textbf{Gift / Resistenz:} & 10 / 9 \\
    \textbf{Psyche / Widerstand:} & 5 / 7 \\
    \textbf{Bewegung:} & 6 \\
    \textbf{Aktionspunkte:} & 4\\
    \textbf{NK-Waffe (Hörner)}: & +1, 1 m
\end{tcolorbox}

\begin{tcolorbox}[title= Eigenschaften, colbacktitle=mybrown, tabulars={@{\extracolsep{\fill}\hspace{1mm}}cccccc@{\hspace{1mm}}}, boxrule=0.5pt]
    \textbf{Rob} & \textbf{Gew} & \textbf{Spi} & \textbf{Ini}  & \textbf{Kra} & \textbf{Ges} \\
    9 & 14 & 7 & 9 & 10 & 12\\ \hline
    \textbf{Int} & \textbf{Cha} & \textbf{Emo} & \textbf{Krea}  & \textbf{Sin} & \textbf{Wil} \\
    8 & 7 & 8 & 4 & 12 & 7
    
\end{tcolorbox}
Wie fleischgewordene Albträume werden die zwei Meter große Hirschwesen beschrieben. Diese scheuen Einzelgänger versuchen Kontakt zu anderen Rassen so gering wie möglich  zu halten. Untereinander sind Rothörner zutrauig und teilen offen Nahrungsquellen, Informationen und vor allem Sagen. Wie faszinierte Kinder lieben die Rothörner Geschichten von mächtigen Horngenossen. In der Regel wird dem Gastgebenden Rothorn die Ehre teil Geschichten zu erzählen. Dafür wird ihm gedankt und ihm wird die Ehre der Wahl des Geschlechtsverkehrspatners für die Nacht offengelegt.
Das Rothorn ernährt sich rein vegetarisch und nutzt seine Krallenfinger um ihre Nahrung aus Vertiefungen zu kratzen.

\subsection*{Fluchtverhalten}
Rothörner kämpfen nie wenn sie keine Chance auf Erfolg sehen. Lieber würden sie sich gefangen nehmen lassen oder, was ihnen am liebsten ist, fliehen. \\
\textbf{Effekt:} \textit{Passiv}. Jedes Mal wenn eine befreundetes Rothorn stirbt macht das Rang höchste Rothorn eine Willenskraftprobe(9+) mit einem Mali von -3 pro gefallenes Rothorn. Sollte sie missglücken fliehen die Rothörner. Wenn das nicht möglich ist versuchen sie sich zu unterwerfen.

\subsection*{Gut vernetzt}
Ein Rothorn läuft selten alleine durch die Wälder. Wenn sie in Panik verfallen rufen sie ihre Artgenossen um eine Chance zu erhalten mit dem Leben davonzu kommen.\\
\textbf{Grundwert:} 7 (Cha)\\
\textbf{Anforderung:} 18 +\\
\textbf{Effekt:} Ein weiteres Rothorn tritt nächste Kampfrunde dem Kampf bei. Kann nur einmal jede Kampfrunde von einem beliebigen Rothorn verwendet werden. Es treten außerhalb ihrer Heimat maximal 3 weitere Rothörner der Kampf bei.

\epigraph{Die Hörner vom Rothorn kann man sich toll an die Wand hämmr'n. Wenn du welche findest natürlich! \textit{Gelächter} Die Drecksdinge sind schnell. Echt schnell. Die sinn sogar dem Mark'i wech gelaufen.}{\textit{Murkat der Steinschmeisser, dorkischer Krieger}}

% _ _ _ _ _ _ _ _ _ _ _ _ _ _ _ _ %

\section*{Schattenvard}

\begin{tcolorbox}[title= Charakteristiken, colbacktitle=mydarkblue, tabulars={@{\extracolsep{\fill}\hspace{1mm}}ll@{\hspace{1mm}}}, boxrule=0.5pt]
    \textbf{Gefahrenstufe:} & 2 \\
    \textbf{Erfahrungspunkte:} & 25 EP \\
    \textbf{Rasse:} & Spirit \\
    \textbf{Größe:} & mittel \\
    \textbf{Wunden:} & 28 \\
    \textbf{Blocken / Ausweichen:} & 6 / 13 \\
    \textbf{Gift / Resistenz:} & 11 / 6 \\
    \textbf{Psyche / Widerstand:} & 12 / 7 \\
    \textbf{Bewegung:} & 6 \\
    \textbf{Aktionspunkte:} & 5 \\
    \textbf{NK-Waffe (Rückengarne):} & +1, 4 m \\
    \textbf{NK-Waffe (Krallen)}: & +2, 0.5 m
\end{tcolorbox}

\begin{tcolorbox}[title= Eigenschaften, colbacktitle=mydarkblue, tabulars={@{\extracolsep{\fill}\hspace{1mm}}cccccc@{\hspace{1mm}}}, boxrule=0.5pt]
    \textbf{Rob} & \textbf{Gew} & \textbf{Spi} & \textbf{Ini}  & \textbf{Kra} & \textbf{Ges} \\
    6 & 13 & 11 & 9 & 5 & 15\\ \hline
    \textbf{Int} & \textbf{Cha} & \textbf{Emo} & \textbf{Krea}  & \textbf{Sin} & \textbf{Wil} \\
    8 & 7 & 8 & 4 & 9 & 7
\end{tcolorbox}

Die schlanke, humanoide Gestalt der Schattenvard ist verbunden mit Krallenfüßen, einem vogelähnlichen Kopf und zwei flache Rückengarnen. Mit schnellen athletischen Manövern ist die Schattenvard in der Lage Gegner in den Rücken zu fallen und sie ausfindig zu machen. Als Vard ist die Schattenvard in der Lage Spirit zu manifestieren und die Beweggründe ihrer Feinde zu erkennen. Mit ihren bis zu zwei Meter langen Rückengarnen kann sie gemeine Schläge ausführen oder ihre Gegner festsetzen. Die Schattenvarde leben häufig als Begleiter von Kultisten der Gri oder im Dienste von gläubigen Dienstherren.

\subsection*{Auslesen}
Als wüsste der Schattenvard exakt was sie tun würde konterte sie die Schläge, wich dem Tritt aus und Schlug überlegen zurück.\\
\textbf{Grundwert:} 11 (Spi) - 4\\
\textbf{Anforderung:} Int[Ziel]+\\
\textbf{Reichweite:} 10 m \\
\textbf{Effekt:} Die Zielkreatur legt seine beste Handkarte ab und zieht eine neue Karte. Falls es keine Spielerkreatur ist, erleidet die Kreatur 5 Psychischen Schaden.

\subsection*{Federnde Sprünge}
Als gäbe es keine Schwerkraft schleudern sich die Schattenvarde von Türmen und Klippen, über Gemäuer und Feinde hinein in das Getümmel.\\
\textbf{Effekt:} \textit{Passiv}. Ignoriere Gelände und Feinde während deiner Bewegung, vorausgesetzt deine Endposition ist frei. Du zählst, solang keine Effekte etwas anderes verlangen, immer als in Bewegung.

\subsection*{Schattenbolzen}
Das Dartförmige Geschoss zieht flatternd einen schwarzen Schweif hinter sich her. Der schwere Aufschlag passt weniger zum grazilen Schuss und errinnert viel ehr an einen nieder schmetternden Hammer. \\
\textbf{Grundwert:} 11 (Spi) + 3\\
\textbf{Anforderung:} 16+\\
\textbf{Reichweite:} 45 m \\
\textbf{Effekt:} \textit{Angriff(Physisch)}. Erleidet ein Bewegtes Ziel Wundenverluste durch Schattenbolzen, so verliert diese Kreatur eine Handkarte/einen Aktionspunkt.

\epigraph{Ich weiß es nicht, meine Herrin. Sie war sehr schnell und sprangen mit Leichtigkeit von Turm zu Turm. Sie liefen die im Schatten der Nacht und ich hatten keine Möglichkeit sie genauer zu betrachten. Als sie bei ihrem Gatten ins Zimmer sprang hörte ich nur noch seine überraschte Stimme. Wie sie abbrach...}{\textit{Larissa Meinla, menschliche Rattenfängerin}}

\subsection*{}

% _ _ _ _ _ _ _ _ _ _ _ _ _ _ _ _ %

\section*{Skriva}

\begin{tcolorbox}[title= Charakteristiken, colbacktitle=mygreen, tabulars={@{\extracolsep{\fill}\hspace{1mm}}ll@{\hspace{1mm}}}, boxrule=0.5pt]
    \textbf{Gefahrenstufe:} & 2 \\
    \textbf{Erfahrungspunkte:} & 5 EP \\
    \textbf{Rasse:} & Kratzer \\
    \textbf{Größe:} & mittel \\
    \textbf{Wunden:} & 32 \\
    \textbf{Schlagen ST/BW:} & 6 (+1) / 12 (+1) \\
    \textbf{Blocken / Ausweichen:} & 6 (+1) / 10 (+1) \\
    \textbf{Gift / Resistenz:} & 14 / 6 \\
    \textbf{Psyche / Widerstand:} & 9 / 5 \\
    \textbf{Bewegung:} & 5 \\
    \textbf{Aktionspunkte:} & 5 \\
    \textbf{NK-Waffe (Krallen)}: & +1, 0.5 m \\
    \textbf{Rüstung (Chitin)}: & +1
\end{tcolorbox}

\begin{tcolorbox}[title= Eigenschaften, colbacktitle=mygreen, tabulars={@{\extracolsep{\fill}\hspace{1mm}}cccccc@{\hspace{1mm}}}, boxrule=0.5pt]
    \textbf{Rob} & \textbf{Gew} & \textbf{Spi} & \textbf{Ini}  & \textbf{Kra} & \textbf{Ges} \\
    6 & 10 & 9 & 11 & 6 & 12 \\ \hline
    \textbf{Int} & \textbf{Cha} & \textbf{Emo} & \textbf{Krea}  & \textbf{Sin} & \textbf{Wil} \\
    6 & 3 & 5 & 3 & 9 & 5
\end{tcolorbox}

\subsection*{Säurespeien}
Als wahrer Allesfresser benötigen die Skriva besondere Säuren um mit den Fremdkörpern klar zu kommen. Wenn diese Säuren jedoch ausgespuckt werden sind sie umso schrecklicher.\\
\textbf{Grundwert:} 12 (Gesch) \\
\textbf{Anforderung:} 18+ \\
\textbf{Effekt:} \textit{Angriff (Gift)}. Reduziere die Rüstung des Ziels um \textbf{FK}. \textbf{Bonus:} Toxin.

\subsection*{Toxinstich}
Mit einer flinken Bewegung sticht der wendige Skriva seinen Klauenbesetzen Schwanz in die Organe seines Gegners. Das übertragende Gift kann nun frei im langsam zerfallenden Körper propagieren.\\
\textbf{Grundwert:} 12 (Gesch) \\
\textbf{Anforderung:} Bewegt \\
\textbf{Reichweite:} 0.75 m \\
\textbf{Effekt:} \textit{Angriff (Gift)}. Sollte eine Kreatur Giftwunden erleiden, so erleidet sie den Effekt deines Giftes. \textbf{Bonus:} Toxin.

\subsection*{Volk der Unterwelt}
Durch ihr Leben in der dunklen Unterwelt haben die Skriva ihren Sehsinn optimiert.\\
\textbf{Effekt:} \textit{Passive.} Du kannst in Dämmerung auf 50 m noch normal sehen und in Dunkelheit 30 m. Allerdings bist du leicht zu blenden.

% _ _ _ _ _ _ _ _ _ _ _ _ _ _ _ _ %

\section*{Skriva (Schwarmführer)}

\begin{tcolorbox}[title= Charakteristiken, colbacktitle=mygreen, tabulars={@{\extracolsep{\fill}\hspace{1mm}}ll@{\hspace{1mm}}}, boxrule=0.5pt]
    \textbf{Gefahrenstufe:} & 3 \\
    \textbf{Erfahrungspunkte:} & 30 EP \\
    \textbf{Rasse:} & Kratzer \\
    \textbf{Größe:} & mittel \\
    \textbf{Wunden:} & 42 \\
    \textbf{Schlagen ST/BW:} & 9 (+2) / 15 (+2) \\
    \textbf{Blocken / Ausweichen:} & 9 (+3) / 13 (+3) \\
    \textbf{Gift / Resistenz:} & 19 / 9 \\
    \textbf{Psyche / Widerstand:} & 14 / 9 \\
    \textbf{Bewegung:} & 5 \\
    \textbf{Aktionspunkte:} & 5 \\
    \textbf{NK-Waffe (Krallen)}: & +2, 0.5 m \\
    \textbf{Rüstung (Chitin)}: & +3
\end{tcolorbox}

\begin{tcolorbox}[title= Eigenschaften, colbacktitle=mygreen, tabulars={@{\extracolsep{\fill}\hspace{1mm}}cccccc@{\hspace{1mm}}}, boxrule=0.5pt]
    \textbf{Rob} & \textbf{Gew} & \textbf{Spi} & \textbf{Ini}  & \textbf{Kra} & \textbf{Ges} \\
    9 & 13 & 12 & 14 & 9 & 15 \\ \hline
    \textbf{Int} & \textbf{Cha} & \textbf{Emo} & \textbf{Krea}  & \textbf{Sin} & \textbf{Wil} \\
    10 & 5 & 8 & 5 & 12 & 9
\end{tcolorbox}

\subsection*{Säurespeien}
Als wahrer Allesfresser benötigen die Skriva besondere Säuren um mit den Fremdkörpern klar zu kommen. Wenn diese Säuren jedoch ausgespuckt werden sind sie umso schrecklicher.\\
\textbf{Grundwert:} 15 (Gesch) \\
\textbf{Anforderung:} 18+ \\
\textbf{Effekt:} \textit{Angriff (Gift)}. Reduziere die Rüstung des Ziels um \textbf{FK}. \textbf{Bonus:} Toxin.

\subsection*{Schattenläufer}
Die Skriva sind seid Generationen schon Kreaturen der Schatten. Auch wenn sie nicht unter offenen Bereichen leiden fühlen sie sich schnell ungedeckt und unwohl. Im Gegenzug sind sie jedoch hervoragende Jäger in den Schatten und unglaublich schwer auszumachen, wenn sie nicht gesehen werden wollen.
\textbf{Grundwert:} 13 (Gew) \\
\textbf{Anforderung:} 18 +\\
\textbf{Effekt:} Du musst dich innerhalb von einem Meter Entfernung von einem Objekt befinden das min. eine Größenortnung größer als du ist. Du tarnst dich und erhälst somit einen Bonus von 8 auf deine Verteidigung. Sobald du Wunden erleidest, eine Aktion außerhalb eines Schatten bzw. im Nahkampf mit einer Kreatur beendest oder selber angreifst, verblasst die Tarnung. Erhalte ein Vorteil.

\subsection*{Toxinwolke}
Das närrische Funkeln der Vorfreude in den Augen eines Skrivas bedeutet seltens etwas gutes. Hört man Sekunden später ein sanftes Zischens, richt schwefelartige Gerüche und sieht ein ungleichmäßig verteiltes Grün aufsteigen, so weiß man dass man definitiv nicht atmen sollte! \\
\textbf{Grundwert:} 12 (Spi) \\
\textbf{Anforderung:} Stationär \& $\Karo{}$ 18 +\\
\textbf{Radius:} 1 m \\
\textbf{Effekt:} Alle Kreaturen innerhalb des Wirkungsradius müssen eine Robustheitsprobe (17+) bestehen oder werden Opfer deines aktiven Gifts.
\textbf{Bonus:} Toxin.

\subsection*{Toxinstich}
Mit einer flinken Bewegung sticht der wendige Skriva seinen Klauenbesetzen Schwanz in die Organe seines Gegners. Das übertragende Gift kann nun frei im langsam zerfallenden Körper propagieren.\\
\textbf{Grundwert:} 15 (Gesch) \\
\textbf{Anforderung:} Bewegt \\
\textbf{Reichweite:} 0.75 m \\
\textbf{Effekt:} \textit{Angriff (Gift)}. Sollte eine Kreatur Giftwunden erleiden, so erleidet sie den Effekt deines Giftes. \textbf{Bonus:} Toxin.

\subsection*{Volk der Unterwelt}
Durch ihr Leben in der dunklen Unterwelt haben die Skriva ihren Sehsinn optimiert.\\
\textbf{Effekt:} \textit{Passive.} Du kannst in Dämmerung auf 50 m noch normal sehen und in Dunkelheit 30 m. Allerdings bist du leicht zu blenden.

% _ _ _ _ _ _ _ _ _ _ _ _ _ _ _ _ %

\section*{Takdisha Kompan} \label{npc:takdisha Kompan}

\begin{tcolorbox}[title= Charakteristiken, colbacktitle=mysilver, tabulars={@{\extracolsep{\fill}\hspace{1mm}}ll@{\hspace{1mm}}}, boxrule=0.5pt]
    \textbf{Gefahrenstufe:} & 1 \\
    \textbf{Erfahrungspunkte:} & 25 EP \\
    \textbf{Rasse:} & Mechanismus \\
    \textbf{Größe:} & klein \\
    \textbf{Wunden:} & 19 \\
    \textbf{Blocken / Ausweichen:} & 10 / 9 \\
    \textbf{Gift / Resistenz:} & 11 / 10 \\
    \textbf{Psyche / Widerstand:} & 9 / 10 \\
    \textbf{Bewegung:} & 4 \\
    \textbf{Aktionspunkte:} & 5 \\
    \textbf{NK-Waffe (Gefährliche Kanten):} & -1, 0.25 m \\
\end{tcolorbox}

\begin{tcolorbox}[title= Eigenschaften, colbacktitle=mydarkblue, tabulars={@{\extracolsep{\fill}\hspace{1mm}}cccccc@{\hspace{1mm}}}, boxrule=0.5pt]
    \textbf{Rob} & \textbf{Gew} & \textbf{Spi} & \textbf{Ini}  & \textbf{Kra} & \textbf{Ges} \\
    10 & 9 & 5 & 7 & 6 & 11\\ \hline
    \textbf{Int} & \textbf{Cha} & \textbf{Emo} & \textbf{Krea}  & \textbf{Sin} & \textbf{Wil} \\
    9 & 12 & 16 & 6 & 10 & 8
\end{tcolorbox}

Freche Münder würden sie als kleine, planlose Maschinen bezeichnen, aber sie sind \textit{mehr}! Diese kleine Konstruktionen von angehende Witchghulen sind Wunderwerker fantastischer Fantasie. Ihre Funkelcores sind aggressiv zusammengepresste Schwarzkerne deren pure Überkonzentration an Spiritszündungen den Minimrot mit Energie überströmen. Der \textit{Takdisha Kompan} erhält eine mittlere Erweiterung oder zwei leichte Erweiterungen.

% _ _ _ _ _ _ _ _ _ _ _ _ _ _ _ _ %

\section*{Terror Hund}

\begin{tcolorbox}[title= Charakteristiken, colbacktitle=myviolet, tabulars={@{\extracolsep{\fill}\hspace{1mm}}ll@{\hspace{1mm}}}, boxrule=0.5pt]
    \textbf{Gefahrenstufe:} & 1 \\
    \textbf{Erfahrungspunkte:} & 5 EP \\
    \textbf{Rasse:} & Wiedergänger \\
    \textbf{Größe:} & klein \\
    \textbf{Wunden:} & 21 \\
    \textbf{Blocken / Ausweichen:} & 5 / 9 \\
    \textbf{Gift / Resistenz:} & 6 / 5 \\
    \textbf{Psyche / Widerstand:} & 6 / 6 \\
    \textbf{Bewegung:} & 6 \\
    \textbf{Aktionspunkte:} & 4 \\
\end{tcolorbox}

\begin{tcolorbox}[title= Eigenschaften, colbacktitle=myviolet, tabulars={@{\extracolsep{\fill}\hspace{1mm}}cccccc@{\hspace{1mm}}}, boxrule=0.5pt]
    \textbf{Rob} & \textbf{Gew} & \textbf{Spi} & \textbf{Ini}  & \textbf{Kra} & \textbf{Ges} \\
    5 & 9 & 3 & 11 & 5 & 8\\ \hline
    \textbf{Int} & \textbf{Cha} & \textbf{Emo} & \textbf{Krea}  & \textbf{Sin} & \textbf{Wil} \\
    7 & 6 & 8 & 5 & 11 & 6
\end{tcolorbox}
Dies sind keine normalen Untoten Hunde, es sind solche die durch spezielle Riten zu Tötungmaschienen gemacht wurden. Sie jagen alles was sich bewegt.\\

\subsection*{Flink}
Die geringe Größe kombiniert mit den schnellen Pfoten ermöglichen dem hungrigen Terrorwolf schnelle Manöver und unvorhergesehene Haken.\\
\textbf{Effekt:} \textit{Passiv}. Kann sich ohne Gegenschläge aus dem Nahkampf entfernen.

\subsection*{Rudelkämpfer}
........\\
\textbf{Effekt:} \textit{Passiv}. Für jeden Verbündeten der sich im Nahkampfreichweite deines Ziels befindet erhalte einen Bonus von +3 auf Angriffe.

% _ _ _ _ _ _ _ _ _ _ _ _ _ _ _ _ %

\section*{Treant}

\begin{tcolorbox}[title= Charakteristiken, colbacktitle=mydarkblue, tabulars={@{\extracolsep{\fill}\hspace{1mm}}ll@{\hspace{1mm}}}, boxrule=0.5pt]
    \textbf{Gefahrenstufe:} & 2 \\
    \textbf{Erfahrungspunkte:} & 20 EP \\
    \textbf{Rasse:} & Spirit \\
    \textbf{Größe:} & groß \\
    \textbf{Wunden:} & 41 \\
    \textbf{Schlagen ST/BW:} & 13 / 7 \\
    \textbf{Blocken / Ausweichen:} & 14 / 7 \\
    \textbf{Gift / Resistenz:} & 15 / 14 \\
    \textbf{Psyche / Widerstand:} & 12 / 5 \\
    \textbf{Bewegung:} & 4 \\
    \textbf{Aktionspunkte:} & 5 \\
\end{tcolorbox}

\begin{tcolorbox}[title= Eigenschaften, colbacktitle=mydarkblue, tabulars={@{\extracolsep{\fill}\hspace{1mm}}cccccc@{\hspace{1mm}}}, boxrule=0.5pt]
    \textbf{Rob} & \textbf{Gew} & \textbf{Spi} & \textbf{Ini}  & \textbf{Kra} & \textbf{Ges} \\
    14 & 7 & 12 & 5 & 13 & 7\\ \hline
    \textbf{Int} & \textbf{Cha} & \textbf{Emo} & \textbf{Krea}  & \textbf{Sin} & \textbf{Wil} \\
    7 & 3 & 6 & 9 & 13 & 5
\end{tcolorbox}
Treants sind Bäume die eine tiefe Verbindung mit dem Spiritfeld haben und zum Leben erwacht sind. 
Da sie ein Band nicht nur unter sich sondern mit allen Bäumen haben merken sie sofort wenn im Wald etwas vor sich geht.\\

\subsection*{Verwurzelt}
........\\
\textbf{Effekt:} \textit{Passiv}. Wenn du stehst bist du verwurzelt und kannst nicht bewegt werden.

\subsection*{Waldbund}
........\\
\textbf{Effekt:} \textit{Passiv}. Verspürt alles was die anderen Bäume in deiner nähe spüren.


