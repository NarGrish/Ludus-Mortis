\chapter{Ausrüstung} \label{Ausrüstung}
In den Weiten der heimlichen Welt werden die Abenteuerer auf Händler, Schätze und frei vererbbaren Überbleibsel von Feinden treffen. Auch wenn man einem Wareguard keinen Rucksack anreichen wird, hat jede Kreatur zugang zu einem eigenen Inventar und der Möglichkeit Gegenstände zu transportieren. Es liegt an der Kreativität der Spieler dem Spielleiter mit nachvollziehbaren Wegen des Transport entgegenzukommen, um ihm zu erläutern wie sie ihre neuen Schätze transportieren wollen.

\section{Währungen}
In den heimlichen Landen wird ein fleißiger Sammler viele verschiedenen Währungen antreffen. Seien es die Groschen des Dolreichs, die "Knochen" der Nurcart oder die Golden der Vampire. Um Tabellenwerke zu vereinfachen nutzen wir ein Einheitswährung \textit{EW} die sich in die anderen Währungen umrechnen lässt. Sollte es die Spielergruppe stören zwischen der verschiedenen Währungen hin und her zuwechseln, kann der Spielleiter auch entscheiden ausschließlich \textit{EW} zu nutzen und so aufwendigere Listen zu umgehen. Andererseits kann es auch zu Stimmung beitragen die verschiedenen Münzen der Völker in der Hand zu haben und so Grenzübergängen eine weitere Tiefe zuzufügen.

\subsection*{Der Dolreichgroschen}
Die wichtigste Währung des Südens ist der Dolreichgroschen. Man wird ihn nicht bloß in den Dolreichen selber antreffen sondern auch in handelnden Nachbarländern antreffen. Diese sehr simple Währung besteht aus zwei Münzentypen: dem Groschen und dem Kelnen (\textit{Kleiner} im alt Delchen). Mit einem Kelnen wird man sich zwar kein Pferd kaufen können, doch falls ihr dem Jungen der euch gerade den Weg gewiesen habt ein Lächeln aufs Gesicht zaubern wollt, so wird es eine dieser zehnkanitgen Bronzemünzen tun. Ein \textbf{Kelnen} entspricht \textbf{fünf EW}. Es sind ebenfalls 20-Kelnen-Münzen im Umlauft deren feingerillter Rand das Schiff der ersten Menschen umrahmt. Da diese Münze jedoch auch für die Unterjochung der Skriva steht, wird sie nicht weiter geprägt und kann in den flaschen Situationen zu Streitigkeiten führen.
Der sehr wertvolle Groschen entspricht der monatlichen Einnahmen eines Soldaten der Reichskraft. Er erhielt bereits früh den Namen der Lohnmünze, obwohl die Soldaten ihre Bezahlung meist in Form von Kelnen bekamen. \textbf{100 Kelnen} entsprechen \textbf{einem Groschen}. Die goldene vierkantige Münze entspricht \textbf{500 EW}.

\subsection*{Die Nurcarter Knochen}
Aus dem hohen kulturellen Wert der Knochen des erlegten wild in der Historie der Nurcarter Zivilisation wird die bleiche Währung der kalten Region in Schädel, Knöchel und Zähne unterteilt.
Die dreieckigen Messingmünzen werden als \textbf{Zähne} bezeichnet und entsprechen einem Wert von \textbf{1 EW}. In der Regel kann gleichwertig auch mit echten Zähnen bezahlt werden, was die vielen zahnlosen Leichen in den alten Rauchkriegen erklärt. Ein vollen Set an Norkzähne (also \textbf{28 Stück}) wiegen \textbf{ein Knöchel} auf. \textbf{Ein Knöchel} entspricht somit \textbf{28 EW} und reicht somit für die großzahl an alltäglichen Besorgungen. Somit reicht ein Knöchel in der Regel für ein solides Abendbrot oder eine Ecke im Stall für die Nacht aus. Reiche Gäste wird man an Schädeln erkennen. \textbf{Zehn Knöchel}, \textbf{280 Zähne} oder \textbf{280 EW} entsprechen \textbf{einem} der goldenen \textbf{Schädeln}. Diese passend geformten Taler Werden gerade auf den Inseln verpöhnt und mit hochnasiger Gesellschaft assoziiert.

% - - - - - - - - - - - - - - %
% - - - - - - - - - - - - - - %

\section{Waffen} \label{waffen}
In den gefährlichen Weiten der heimlichen Landen finden sich unzählige Gründe ein Waffe zu tragen. Waffen bringen ihrem Führer erhöhte Reichweiten, einen Boni auf \textit{Schlagenaktionen} und teilweis einen weiteren Effekt ein. Sie können in verschiedene Kategorien eingeteilt werden:\\

\subsection*{Nahkampfwaffen}
Einhändige Nahkampfwaffen können eine Distanz von (Kraft/2) m weit geworfen werden und bekommen teilweise in eckigen Klammern Boni.

\subsubsection*{Hiebwaffen (1H)} \label{ar:hiebwaffe}
Unter die Kategorie der Hiebwaffen fallen die meisten einhändigen Waffen. Seien es nun Schwerter, Äxte und leichte Keulen.\\
\textbf{Kampfaktion:} \textit{Schlagen}\\
\textbf{Kosten:} 50 bis 1000 EW\\
\textbf{Bonus:} +0 bis +3, [+0 bis +1]\\
\textbf{Reichweite:} 1 m

\subsubsection*{Hieblanze (1H)} \label{ar:hieblanze}
Die 2 Meter große Axtähnliche Waffe der Nurcarter Mrots wurde von den großen Schmieden der Norks kreiert und hergestellt. Ihr Zahnradschiene ermöglicht es dem Mrot eine bewegliche Verbindung unterhalb der Hand an der Waffe zu befestigen um die Hieblanze auch einhändig verwenden zu können. An beiden Enden der Stabwaffe regt eine schmale Klinge hervor um Schwachstellen anzuvisieren und einem Gegner endgültig den Gar aus zu machen. \\
\textbf{Kampfaktion:} \textit{Schlagen}\\
\textbf{Kosten:} 3000 EW\\
\textbf{Bonus:} +4\\
\textbf{Reichweite:} 2.25 m \\
\textbf{Effekt:} Du darfst zu Beginn einer Schlagen Aktion ansagen deinen Gegner \textit{aufzuspießen}. Sowohl dein Waffenbonus, als auch all seine Rüstung- und andersartigen Verteidigungsbonis verfallen. Du erhälst in beiden Fällen den \textbf{Bonus} \textit{Element}.\\
Sollte ein Nicht Mrot diese Waffe nutzen wollen so ist sie ein Zweihänder und gibt nicht den \textit{Element}-Boni.

\subsubsection*{Pickeln (1H)}
Die Grauskriva der hohen Nav'rifberge sind bekannt für ihre brutalen Pickelkampfkünste. Die häufig in Paaren geführten Waffen sind besonders für schleichende Angriff hervoragend geeignet.\\
\textbf{Kampfaktion:} \textit{Schlagen}\\
\textbf{Kosten:} 300 bis 500 EW\\
\textbf{Bonus:} +2 bis +3\\
\textbf{Reichweite:} 0.75 m\\
\textbf{Effekt:} Schleichende Angriffe mit Picken erhalten ein zusätzlichen Bonus von +2.

\subsubsection{Sense (2H)}
Die Sense ist eines der offensichtlichsten Werkzeuge die zur Waffe umfunktioniert werden. Ihr Gefährlichkeit rührt vor allem aus der großen Fläche die sie abdecken können.\\
\textbf{Kampfaktion:} \textit{Schlagen}, \textit{\nameref{sk:niederdreschen}}\\
\textbf{Kosten:} 80 EW\\
\textbf{Bonus:} +1 \\
\textbf{Reichweite:} 1.25 m

\subsubsection*{Speere (1H oder 2H)} \label{ar:speere}
Den schlichten Speer findet man in der Hand der meisten einberufenden Soldaten. Er ist simpel, günstig und effektiv. Seine Reichweite schafft ihm zusätzlich ein Vorteil gegenüber den üblichen Hiebwaffen.\\
\textbf{Kampfaktion:} \textit{Schlagen} \\
\textbf{Kosten:} 90 EW\\
\textbf{Bonus:} +1, [+2]\\
\textbf{Reichweite:} 1.75 m\\
\textbf{Effekt:} Wird der Speer mit zwei Händen geführt so hat der Spieler Zugang zu \textit{\nameref{sk:Speerwall}}.

\begin{figure}[htbp]
		\includegraphics[width=8cm]{Pictures/Sticharm.png}
        \label{fig:Sticharm}
\end{figure}

\subsubsection*{Stichhänder (2H)} \label{ar:stichhaender}
Die wohl berühmteste Schwertart ist der Stichhänder der Nork. Diese kunstvollen grazilen Langschwerter sind als Mischung aus einem Speer und einem Schwert besonders für die Jagd auf große Kreaturen konzipiert worden.\\
\textbf{Kampfaktion:} \textit{Schlagen} \\
\textbf{Kosten:} 1500-3000 EW\\
\textbf{Bonus:} +2\\
\textbf{Reichweite:} 2 m\\
\textbf{Effekt:} Pro Größenkategorie des Gegners gibt es ein Boni/Mali von +/-2 beginnt mit +0 bei mittelgroßen Kreaturen. 

% - - - - - - - - - - - - - - %

\subsection*{Fernkampfwaffen}
Fernkampfwaffen dürfen im Nahkampf als Improvisierte Waffe benutzt werden. Sollten sie ein Bonus geben so ist dies in eckigen Klammern angegeben. Einige Waffen nutzen \textit{Munition}. Eine Spielergruppe sollte sich vor dem Spiel einigen ob z.B. Bogenschützen Listen über ihre Pfeile führen sollten oder zum Gunsten des Spielflusses dies umgangen werden soll, in dem man annimmt das der Spieler ausreichend Pfeile mit genommen hat. Munitionswaffen verschießen diese und verweilen somit in den Händen des Kämpfers. Alle anderen Fernkampfwaffen werden weggeworfen und müssen somit erst wieder eingesammelt werden.

\subsubsection*{Armbrust (2H)}
Die Armbrust wird gerne von dolreichern Soldaten und vampirischen Piraten verwendet. Ihre einfache Handhabung ermöglicht einen einfachen Umgang, mit großen Schadenspotential.\\
\textbf{Kampfaktion:} \textit{Schießen}\\
\textbf{Kosten:} 400 bis 1000 EW\\
\textbf{Bonus:} +4 bis +8, [+1]\\
\textbf{Reichweite:} 80 bis 150 m\\
\textbf{Effekt:} Munition. Die Armbrust muss nach jedem Schuss nachgeladen werden. Diese Bonusaktion kann nicht im Nahkampf ausgeführt werden.

\subsubsection*{Bogen (2H)} \label{ar:bogen}
Der Bogen ist die Waffe die am häufigsten mit dem Fernkampf verbunden wird. Je nach Ausführung und der Stärke des Schützen kann er immense Entfernungen überbrücken.\\
\textbf{Kampfaktion:} \textit{Schießen}\\
\textbf{Kosten:} 150 bis 450 EW\\
\textbf{Bonus:} +0 bis +3\\
\textbf{Reichweite:} 80 bis 110 m  PLUS (Kraft * 5) m\\
\textbf{Effekt:} Sollte mit Bögen geschlagen werden zerbrechen sie falls der Schaden mit einem Pick erfolgreich verteidigt wurde (Kein Wundenverlust).

\subsubsection*{Cnacka (1H)} \label{ar:cnacka}
\textbf{Kampfaktion:} \textit{Werfen}\\
\textbf{Kosten:} 30 EW\\
\textbf{Reichweite:} 20 m PLUS (Kraft * 5) m\\
\textbf{Effekt:} Wo auch immer der Cnacka aufschlägt wird eine Karte aufgedeckt. Bei einem Zahlenwert verlieren alle Kreaturen innerhalb von 1 m Radius dem Kartenwert entsprechend Wunden. Bei schwarzen Bildern erleiden alle Kreaturen in 1 m  Radius 10 psychische Wunden und sind \textit{\nameref{ef:geschockt}}. Das Aufdecken eines roten Bildes verteilt an alle Kreaturen in 2 m Radius das Leiden \textit{\nameref{ef:brennend}(3)}

\subsubsection*{Gebärt (2H)} \label{ar:gebärt}
Die verrückten Konstruktionen der Orks die sie Gebärte nennen, sind effektiv Scharfschützengewehre der ganz wilden Art. Sie klappern und kreischen, qualmen und verziehen sich, und doch behauptet der Ork sein Meisterwerk in Griff zu haben.\\
\textbf{Kampfaktion:} \textit{Schießen}\\
\textbf{Kosten:} 200 EW\\
\textbf{Bonus:} +(Intelligenz/5)
\textbf{Reichweite:} 60 m\\
\textbf{Effekt:} Gebärte müssen nach jedem Schuss umständlich Nachgeladen werden. Dies wird dir eine Aktion kosten.

\subsubsection*{Wurfscheiben (1H)} \label{ar:wurfscheiben}
Die scharfgeschliffenen Wurfscheiben sind sehr leichte Metallscheiben die surrend durch die Luft schießen um sich in ihre überraschten Opfer zu schneiden.\\
\textbf{Kampfaktion:} \textit{Werfen}\\
\textbf{Kosten:} 50 EW\\
\textbf{Bonus:} +2\\
\textbf{Reichweite:} Kraft + Geschicklichkeit m\\
\textbf{Effekt:} Rüstungswerte werden doppelt gegen diese Waffe angerechnet.

% - - - - - - - - - - - - - - %

\subsection*{Besondere Waffen}
Zu besonderen Waffen zählen alle möglichen Konstrukt die nicht in die oberen Kategorien fallen. Sie können mit einem in eckigen Klammern stehenden Bonus im Nahkampf verwendet werden.

\subsubsection*{Nekromantenstab} \label{ar:nekromantenstab}
Man kann kaum von den Nekromantenstäben sprechen, da sie alle einzigartig aussehen. Mal sind es mit Lungen überzogene Walrippen, mal stahlgeschmiedete Schönheiten. Doch sie alle sind zu einem Zweck in der Hand ihres Besitzers: Als Zeichen der Kontrolle über den Tod.\\
\textbf{Kampfaktion:} \textit{Spirit}-Fähigkeiten\\
\textbf{Kosten:} unbestimmbar\\
\textbf{Bonus:} +1 bis +5, [+1 bis +3]\\\\
\textbf{Effekt:} Sollten Tote oder Untote Kreaturen dem Nekromanten folgen, so tun sie dies nur sollange keine andere Person die Kontrolle über den Stab erlangt. Ansonsten folgen sie den Befehlen des neuen Meisters.

% - - - - - - - - - - - - - - %
% - - - - - - - - - - - - - - %

\section{Kleidung und Rüstung} \label{ent:kleidungundrüstung}

\begin{tcolorbox}[title= Rüstung ,colbacktitle=gray, tabulars={@{\extracolsep{\fill}\hspace{5mm}}c|c|c@{\hspace{1mm}}}, boxrule=0.5pt]
    Kettenhemd & +3 &  500 EW \\
    Schuppen Panzer & +2 & 200 EW \\
    Steppwams  & +1 & 50 EW \\
    Plattelpanzer & +4 & 1100 EW 
\end{tcolorbox}
\vspace*{0.4 cm}

\begin{tcolorbox}[title= Kleidung ,colbacktitle=gray, tabulars={@{\extracolsep{\fill}\hspace{5mm}}c|l@{\hspace{1mm}}}, boxrule=0.5pt]
    Damenstiefel & 35 EW \\
    Handschuhe & 25 EW \\
    Hemd [E] & 40 EW \\
    Lederne Hose [E] & 35 EW \\
    Mantel & 45 EW \\
    Strohhut & 20 EW \\
    Wanderstiefel [E] & 40 EW
\end{tcolorbox}
\vspace*{0.4 cm}


% - - - - - - - - - - - - - - %
% - - - - - - - - - - - - - - %

\section{Abenteuer Ausrüstung} \label{ent:abenteuerausrüstung}

\subsection*{Reiseausrüstung}

\begin{tcolorbox}[title= Reiseausrüstung,colbacktitle=gray, tabulars={@{\extracolsep{\fill}\hspace{5mm}}c|c@{\hspace{1mm}}}, boxrule=0.5pt]
    Decke & 40 EW \\
    Rucksack & 60 EW \\
    Wanderstab [+0,1 m] & 30 EW \\
    1-Mann-Zelt & 85 EW
\end{tcolorbox}
\vspace*{0.4 cm}

\subsubsection*{Feldflasche / Trinkschlauch} \label{ar:feldflaschetrinkschlauch}
Es ist essenziell am Tag genug zu trinken. \\
\textbf{Kosten:} 10 bis 60 EW, Nachfüllen:
\begin{itemize}
    \item \textbf{Wasser:} 2 EW
    \item \textbf{Bier:} 5 EW
    \item \textbf{Wein:} 10 EW
\end{itemize}
\textbf{Effekt:} Der Inhalt einer \nameref{ar:feldflaschetrinkschlauch} reicht für zwei Tage in der Wildnis. Sollte er am dritten nicht aufgefüllt oder ersetzt worden sein, so erhält die Kreatur ein Malus von -3 auf alle Eigenschaften. Dieser Effekt wiederholt sich mit jedem weiteren Tag bis eine Eigenschaft auf/unter 0 fällt und die Kreatur somit stirbt.

\subsubsection*{Landkarte}
Um sich schnell und effektiv durch unbekannte Gebiete bewegen zu können benötigt es häufig Landkarten. Je nach der Region können diese leicht zu erhalten oder in unbekannten Gefilden gar unmöglich aufzufinden sein. \\
\textbf{Kosten:} 30 bis 100 EW \\
\textbf{Effekt:} Der Besitz einer Karte sollte den Spielern einen Zugang zu den Skizze des Landes durch den Spielleiter ermöglichen. Zu dem schaffen reisende Kreaturen an einem Tag eine Distanz hin zu legen die einer Reisedauer von einer Stunde mehr entspricht.

\subsection*{Skailen}
Die Skailen sind mechanische Wunderwerke des Ostens. Sie entstanden aus den Kooperationen zwischen den talentierten Skriva und kreativen Menschen, wärend einer Zeit in der Reichtum und Luxus mehr bedeuteten als lediglich das materielle da sein, sondern auch ein Ziel und eine Perspektive dar stellten. Skailen werden in der Regel durch Federn, Schwinggewichte oder chemische Reaktionen angetrieben. Sie basieren nahezu nie auf Spiritmanifestationen und können so überall eingesetzt werden. Da es tausende von verschiedene Versionen der handgearbeiteten Stücke gibt, werden hier nur ein paar der beliebten Grundskailen auf geführt und es liegt am Erzähler sich weitere auszudenken oder diese hier zu modifizieren.

\subsubsection*{Narkta}
Narktas sind eine Form von Uhren. Im Gegen zu vielen anderen Zeitkonzepten skalieren die Zeiteinheiten der Nietküste mit der Uhrzeit. So wird die Narkta im späten Abend langsamer rotieren als im frühen Morgen. In der Regel wird der Benutzer die Uhr mit dem Sonnenaufgang neu starten müssen, da zum einen Null Uhr mit der Sonnenaufgang genormt ist und zum Anderen die antreibenden Resourcen wie eine Feder neu gespannt werden müssen. Narktas sind zwischen 200 und 700 EW wert.

% - - - - - - - - - - - - - - %

\section{Freizeit Ausrüstung}

\subsubsection*{Zirhaar}
Das Instrument der Narcarter Norks steht in einem kontroversen Widerspruch mit den Vorwürfen der Südländer das der Nork ein herzloses, barbarisches Wesen sei, das nur für den Krieg existiere. Denn auch wenn dass noch nicht einem für seine brutaleren Artgenossen stimmt, ist der Nork sehr tiefsinnig und gesellig. Wenn er zusammen mit seinen Kameraden am abendlichen Feuer sitzt und der harte Arbeit einen weiteren Tag gestrotzt hat dünstet es ihn nach einer leichten und zugleich zarten Musik. Das Zirhaar besteht aus einem sauber geschnitzen Horn von z.B. Rothörnern und Saiten von verschiedensten Tieren. Am unteren Ende des Zirhaars befindet sich eine gespannte Tierhaut, so dass der Spieler während des Zupfens mit seiner Handfläche oder dem Daumen ein ryhtmisches Schlagen erzeugen kann. \\
\textbf{Kosten:} 500 - 1500 EW

% - - - - - - - - - - - - - - %

\subsection*{Spirituele Ausrüstung}

\subsubsection*{Bund der Schritte} \label{ar:bundderschritte}
In der Zeremonie der Wegscheide wird dem Draekolin dieses Stoffbund bestückt mit Runen überreicht. Mit dem Wandel der Zeit sammelt der Draekolin Erinnerungen angesteckt an seinem Bund um sich stets seinem gelaufenen Weg bewusst zu sein um nach Hause zurückfinden zu können.
\textbf{Kosten:} 55 EW\\
\textbf{Effekt:} Wähle ein jedes Mal wenn du eine Pfadausprägung wählen darfst eine kleine Erinnerung aus deinem Umfeld, dass du ein Bund bindest. Solltest du sterben und den Bund bei dir tragen darfst du 5 der Erinnerungen Opfern um den Pfad zurück zu gehen und deine Wunden zu heilen. Dieser Effekt tritt nach ca. einer viertel Stunde ein.

\subsubsection*{Seelenfänger} \label{ar:seelenfänger}
Seelenfänger sind Ruß-Lehm-Gefäße die von herumziehenden Nekromanten verkauft und genutzt werden. Sie können die Seelen von sterbenden aufnehmen und so einem geschickten Nekromanten erlauben den Geist der Seele zu erwecken.\\
\textbf{Kosten:} 200-1000 EW\\
\textbf{Effekt:} Die erste sterbende Kreatur die diesen Behälter berührt gibt ihre Seele an den Behälter ab. Es besteht an sich keine Möglichkeit mit den gefangenen Seele zu interagieren. Mit der Zerstörung des Seelenfängers kann die Seele ihrem Gefängnis entfliehen.

% - - - - - - - - - - - - - - %
% - - - - - - - - - - - - - - %

\section{Bestandteile der Mrots} \label{ent:mrots}

\subsection*{Untersützungs-Cores} \label{core}
Um mehr Rechenleistung aus einem Mrots zu hohlen wurden Unterstützungs-Cores entwickelt. Sie bassierten auf einfachen KIs und sollen dem Mrots einfache Aufgaben abnehmen. Ein Unterstützungs-Core wäre aber niemals in der lage sich zu einem Mrots zu entwickeln.

\subsubsection*{Supcore}\label{Supcore}
Als Unterstützungscore wird der Sup core vorallem in MK-I Modellen verwendet um deren besonderes Doppelcoremodell auszunutzen.\\
\textbf{Kosten:} 2000 EW\\
\textbf{Energiebeitrag:} 8 \\
\textbf{Besonderheiten:} +1 auf alle Körperattribute

% = = = = = = = = = = = = = = = = = = = %

\subsection*{Strukturen}

\subsubsection*{MK-I Powerbox}
Die wohl am meisten vertretenste Struktur stellt die Boombox dar. Ihre besondere Doppelcoretechnologie ermöglicht es dem Mrot seinen eigenen Core auszutauschen, besonders flexibel zu sein und vorallem ganz viel Energie freizusetzen.\\
\textbf{Kosten:} 2000 EW\\
\textbf{Bewegung:} 4 \\
\textbf{Größe:} Mittel \\
\textbf{Arme:} 2 \\
\textbf{Beine:} 2 \\
\textbf{Erweiterungen:} \begin{itemize}
	    \item \textbf{Heavy:} 2
	    \item \textbf{Medium:} 2
	    \item \textbf{Light:} 1 \end{itemize}
\textbf{Besonderheiten:} Bis zu zwei Kerne.

\subsubsection*{MK-II Ranna}
Die MK-II Entwicklung hatte eine günstige vielseitig einsetzbare Maschine zu entwickeln, die leicht in großer Zahl produzierbar sein sollten.\\
\textbf{Kosten:} 1700 EW\\
\textbf{Bewegung:} 6 \\
\textbf{Größe:} Mittel \\
\textbf{Arme:} 3 \\
\textbf{Beine:} 2 \\
\textbf{Erweiterungen:} \begin{itemize}
	    \item \textbf{Medium:} 4
	    \item \textbf{Light:} 3 \end{itemize}
\textbf{Besonderheiten:} +1 auf alle Attribute.

\subsubsection*{MK-III Bietz}
Mit der Entwicklung von den Ultra-Waffen aus dem Hause Kappnert brach eine neue Ära der großen Waffen an. Um diese in die Schlacht tragen zu können wurde der MK-III entwickelt, welcher auf seinem Rücken ein super heavy Modul tragen kann.\\
\textbf{Kosten:} 3100 EW\\
\textbf{Bewegung:} 6 \\
\textbf{Größe:} Groß \\
\textbf{Arme:} 2 \\
\textbf{Beine:} 4 \\
\textbf{Erweiterungen:} \begin{itemize}
        \item \textbf{Super Heavy:} 1
	    \item \textbf{Medium:} 1
	    \item \textbf{Light:} 2 \end{itemize}
\textbf{Besonderheiten:} Das Bietz darf in seiner Bewegung schwieriges Gelände ignorieren.

\subsubsection*{MK-IV Fliesha}
Die Invasionskriege der Karlvenanlianz zwang die Orks dazu einen Weg zu finden den Kampf in die Lüfte zu tragen um ihren Gegnern ihren größten Vorteil abzunehmen. Nach einigen gescheiterten Prototypen wurde der MK-IV entwickelt. Die weiten, kantigen Flügel sind das Markenzeichen des fliegenden Konstrukts, welches aufgrund der Schwerkraft nur leichtbepackt werden kann. \\
\textbf{Kosten:} 2300 EW\\
\textbf{Bewegung:} 5 \\
\textbf{Größe:} Mittel \\
\textbf{Arme:} 2 \\
\textbf{Beine:} 3 \\
\textbf{Erweiterungen:} \begin{itemize}
	    \item \textbf{Medium:} 2
	    \item \textbf{Light:} 2 \end{itemize}
\textbf{Besonderheiten:} MK-IV Strukturen verfügen über die Fähigkeit \textit{\nameref{sk:erheben}}.


% = = = = = = = = = = = = = = = %

\subsection*{Erweiterungen: Light}

\subsubsection*{Emo-G}
Das Emo-G Modul ist eine Entwicklung des Häckler-Klans um ihre Mrots in Handelszügen mit den Dolreichern schicken zu können. Dieses kleine Kontrollmodul simuliert Gefühlsregungen und unterstützt somit soziale Interaktionen. \\
\textbf{Kosten:} 680 EW\\
\textbf{Energiebedarf:} 2 \\
\textbf{Besonderheiten:} + 2 Charisma

\subsubsection*{High-Tex-Senzor}\label{entw:high-tex}
Mit der Hilfe der Witchgoule ist es den Orks gelungen futuristische Sensor zu konstruieren, mit denen sich Ziele leichter identifizieren lassen. \\
\textbf{Kosten:} 1800 EW\\
\textbf{Energiebedarf:} 3 \\
\textbf{Besonderheiten:} + 1 Sinne und \textit{\nameref{sk:scannen}}

\subsubsection*{Infrasicht} \label{entw:infrasicht}
Das Infrasichtmodul erlaubt es seinem Träger nicht nur Photonen im menschlich sichtbaren Bereich zu Detektieren, sondern auch im Infrarotbereich. Diese ermöglicht auch in der Nacht die ein oder andere lebende Kreatur aufzuspüren.\\
\textbf{Kosten:} 1100 EW\\
\textbf{Energiebedarf:} 3 \\
\textbf{Besonderheiten:} Du kannst auch in der Dunkelheit Lebewesen aus 40 m Distanz erkennen. 

\subsubsection*{Ramhörner}\label{entw:ramhörner}
Die den Minotauren nachgeahnten Ramhörner sind stählerne Waffen die geraden beweglichen Angreifern eine zusätzlichen Vorteil in den Kampf mitbringen. \\
\textbf{Kosten:} 500 EW \\
\textbf{Energiebedarf:} 0 \\
\textbf{Besonderheiten:} In der Runde in der du eine feindliche Kreatur in einen Nahkampf bindest darfst du kostenfrei nach deiner Aktion eine Schlagenaktion ohne Waffenboni mit der obersten Deckkarte ausführen.

\subsection*{Erweiterungen: Medium}

\subsubsection*{Blizbox}
Nicht selten erwartet die mutigen Feinde der Ghule eine warm glitzende Blizbox welche sie in ihrer eigenen Rüstung grillt. \\
\textbf{Kosten:} 2000 EW\\
\textbf{Energiebedarf:} 6 \\
\textbf{Besonderheiten:} \textit{\nameref{sk:blitzentladung}} und \textit{\nameref{sk:schockladung}}

\subsubsection*{Tan-Dinx}
Wie ein Graelfar verschwimmen die riesigen Strukturen des Maschinenmonsters plötzlich mit seinem Umfeld. Man könnte schon fast erraten wem die Orks diesen Trick abgeguckt haben. \\
\textbf{Kosten:} 2750 EW\\
\textbf{Energiebedarf:} 3 \\
\textbf{Besonderheiten:} + 1 Gewandheit und \textit{\nameref{sk:tarnen}}

\subsubsection*{Truhe}
Mrots haben die Angewohnheit keine Ahnung zu haben wo sie Dinge verstauen sollen. Eine einfache eingebaute Truhe löst das Problem. \\
\textbf{Kosten:} 50 EW\\
\textbf{Besonderheiten:} Du kannst Gegenstände verstauen. Wachen wird nicht auffallen, dass dieses Versteck existiert.

\subsubsection*{Dekka-Shield}
Das Knistern eines Dekka-Shields kann so schön sein wie das Zwischern der vorbeifliegenden Vögel. Und deutlich beeindruckender. \\
\textbf{Kosten:} 2600 EW\\
\textbf{Energiebedarf:} 4 \\
\textbf{Besonderheiten:} + 2 Robustheit

\subsubsection*{V-Bow}
Die integrierte V-Bow Einheit der Killa-Serie aus 153 m.Z. wurde nach dem Fall des Grush-Klans aus den besiegten Mrots ausgebaut und unter der Führung der \textit{Krieg \& Schneider Gruppe} als Vorlage einer nun weit verbreiteten Waffe der Mrots genutzt. Dieser einfache Armbrust-ähnlicher Mechanismus, reißt die beiden Enden des Riemens auseinander und schießt so den Bolzen nach Vorne aus der Öffnung. Sobald sie sich zurück bewegen greifen die Zahnradschienen und zwingen die Bolzenhalterung zusammen mit dem Riemen zurück. Mit dem Erreichen des Haltemechanismus, drückt die Bolzenhalterung die Sicherung des nächsten Bolzen weg und er kann nachfallen. Dieser schnelle, vertrauenswürdige Mechanismus bringt den Nachteil eines geringen Schadenspotentials mit sich, ist dafür aber leicht zu handhaben. \\
\textbf{Kosten:} 1100 EW\\
\textbf{Energiebedarf:} 2 \\
\textbf{Bonus:} +1
\textbf{Reichweite:} 80 m\\
\textbf{Besonderheiten:} \textit{schießen}. Für jeden weiteren synchron schießenden V-Bow wird der Bonus verdoppelt (Es kann maximal ein Wert von +16 erreicht werden).

\subsection*{Erweiterungen: Heavy}

\subsubsection*{Greifa}
Den Entwicklern des MK-III fiel einwenig zu spät auf das ihr Mrot gar keine Arme hat. Kurzer Hand konstruierten sie daraufhin Greifa die in das Mediummodule montiert werden konnten. \\
\textbf{Kosten:} 1500 EW\\
\textbf{Energiebedarf:} 4 \\
\textbf{Besonderheiten:} + 1 Arm

\subsubsection*{Portal-Büxxe}
Auch wenn diese Energiereißertechnologie nach dem Halbzahnvorfall verboten wurden, gibt es noch immer einige dieser riskanten Teleporter auf dem Markt. \\
\textbf{Kosten:} 5000 EW\\
\textbf{Energiebedarf:} 8 \\
\textbf{Besonderheiten:} \textit{\nameref{sk:portal}}

\subsubsection*{Burna}
Schon mal gefragt wo her das Sprichwort \textit{etwas sei der Burna} kommt? Nun als die Orks erstaunt ihre Mrots fragte was dieser mächtige Flammenwand war, mit der sie eben die feinde eingeäschert hatten, antworteten die bloß: Das war der Burna! \\
\textbf{Kosten:} 2500 EW\\
\textbf{Energiebedarf:} 6 \\
\textbf{Besonderheiten:} \textit{\nameref{sk:flammeninferno}}

\subsection*{Erweiterungen: Super Heavy}

\subsubsection*{Ultra-Burna}
Nach der Einführung des Ultra-Burnas entwickelte sich eine einzigartige Grilltradition unter den nördlichen Klans unter dem Namen BBQ. \\
\textbf{Kosten:} 4100 EW\\
\textbf{Energiebedarf:} 11 \\
\textbf{Besonderheiten:} \textit{\nameref{sk:flammeninferno}} und \textit{\nameref{sk:bbq}}