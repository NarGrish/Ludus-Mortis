\chapter{Giebswacht} \label{ch:giebswacht}

\section{Eine Führung durch die Bogenstadt}

\subsection*{1: Das Tor}
\subsection*{2: Das andere Tor}
\subsection*{3: Die Wachtburg}

Als das freie Sortiment 299 t.Z. in der Heibschlucht Schutz vor den Stürmen suchten, gründete Oberstjäger Gernkalk Marin hier den Standpunkt Bogenwacht. Wie es für ein Sortimentsstandpunkt üblich ist, konstruierten die Minotauren die Steinernde Wachtburg auf einem zentral liegenden Hügel. Die Festungsanlage bestand ursprünglich nur aus der oberen Hauptburg. Erst später erbauten die Krieger unter dem Kommando von Oberstwache Vahlflut Giebs zwei Jahre vor der großen Belagerung den unter Anbau der Burg.

\section{Die Geschichte der Stadt}

\subsection*{299 t.Z.}
Das freie Sortiment gründet den Standpunkt Bogenwacht.
\subsection*{7 t.Z.}
Die Festungsanlagen werden als Reaktion auf die Nur Carter Bewaffnung erweitert. Zum ersten Mal ist es auch Menschen und Draekolin gestattet den Sortiments Zusatzkräften beizutreten.\\
\subsection*{5 t.Z.}
Nach den schnellen Ansturm der Nur Carter Truppen nehmen sie in einer 3 tägigen Belagerung in einem geschickten Flankenangriff die Stadt ein. Die Wachtburg wird zum neuen Sitz des Warscreamers Malfurk Deark der mit eisernen Hand die Stadt wieder aufbaut. eine Ära der Unterdrückung beginnt, die bis heute nicht als beendet gesehen werden kann.\\